\documentclass[10pt]{article}

\usepackage{framed,fancybox}
\usepackage{multicol,verbatim,overcite,psfrag}
% \usepackage[pdftex,colorlinks]{hyperref}
\usepackage{hyperref}
\hypersetup{%
colorlinks=true,
linkcolor=black,
urlcolor=cyan
}%
% \usepackage{palatino}
\usepackage{epsfig,enumerate,amsmath,amsfonts,latexsym,graphics,graphicx,theorem,graphics}
\usepackage{fancyhdr,subfig}
\usepackage{makeidx,amssymb,longtable}
\newcommand{\hs}[1]{\hspace*{ #1 mm}}
% \renewcommand{\bibname}{References}
% \renewcommand{\theequation}{\mbox{\thechapter--\arabic{equation}}}

\theorembodyfont{\upshape}

\pagestyle{fancy}
% \renewcommand{\chaptermark}[1]{\markboth{Chapter \thechapter{.}\: #1}{}}
% \renewcommand{\sectionmark}[1]{\markright{\thesection\ #1}}
\fancyhf{}
\fancyhead[LE,RO]{\small\bfseries\thepage}
\fancyhead[LO]{\small\bfseries\rightmark}
\fancyhead[RE]{\small\bfseries\leftmark}
\fancypagestyle{plain}{%
\fancyhead{} % get rid of headers
\renewcommand{\headrulewidth}{0pt} % and the line
}

\title{{\bf Quick Reference Guide for $\mathbb{GPOPS-II}$ Version 1.0:} \vspace{12pt}\\
  {\bf A General-Purpose MATLAB Toolbox for Solving Optimal Control Problems
    Using Sparse Nonlinear Programming and $hp$--Adaptive \\ Pseudospectral Methods}} 
\author{Michael A.~Patterson \\ Anil V.~Rao \\\\ {\em University of
    Florida} \\ {\em Gainesville, FL 32611-6250} \\ {\em USA}
  \vspace{24pt} \\ January 2013}
\date{}

\oddsidemargin=0in
\evensidemargin=0in
\topmargin=1in
\hoffset=0in
\voffset=-1.5in
\textheight=9in
\textwidth=6.5in

\headwidth=\textwidth
\renewcommand{\headrulewidth}{0.25pt}
\raggedbottom


% \newcounter{example}[chapter]
% \newcounter{question}[chapter]
% {\theoremstyle{break}\theorembodyfont{\upshape}\newtheorem{example}{Example}[chapter]}
% {\theoremstyle{break}\theorembodyfont{\upshape}\newtheorem{solution}{Solution to Example}[chapter]}
% {\theoremstyle{plain}\theoremheaderfont{\normalsize\bfseries}\theorembodyfont{\upshape}\newtheorem{question}{\hspace{-0.25em}}[chapter]}

% \newcommand{\examplenumber}{\thechapter--\theexample}
% \renewcommand{\theexample}{\thechapter--\arabic{example}}
% \renewcommand{\thesolution}{\thechapter--\arabic{solution}}
% \renewcommand{\thequestion}{\thechapter--\arabic{question}}

%\newcommand{\ecaption}[1]{\addcontentsline{loe}{example}{\protect\numberline{\theexample}#1}}

% \renewcommand{\theequation}{\thechapter-\arabic{equation}}

\usepackage{color}

\definecolor{shadecolor}{gray}{0.99}
\FrameRule=0.75pt
\FrameSep=5pt
\setlength{\fboxrule}{\FrameRule}
\setlength{\fboxsep}{\FrameSep}

\newenvironment{ovalframe}{%
  \cornersize*{20pt}%
  \setlength{\fboxsep}{6pt}%
  \def\FrameCommand{\ovalbox}%
  \MakeFramed{\advance\hsize-\width \FrameRestore}}%
{\endMakeFramed}

\newenvironment{shadedframe}{%
  \def\FrameCommand{\fcolorbox{black}{shadecolor}}%
%  \MakeFramed {\addtolength{\hsize}{-\width}\FrameRestore}}
  \MakeFramed {\FrameRestore}}
{\endMakeFramed}

\newcommand{\bfblue}[1]{\textrm{{\color{blue}{\bf #1}}}}
\newcommand{\slred}[1]{\textrm{\color{red}{\sl #1}}}

\makeindex

\begin{document}

\setcounter{tocdepth}{1}

\newcommand{\mygreekbold}{\boldsymbol}
\newcommand{\mybold}{\mathbf}
%\newcommand{\mybold}{\boldsymbol}
% \newcommand{\mybold}{\textbf}
\newcommand{\Ex}{\textbf{E}_x}
\newcommand{\Ey}{\textbf{E}_y}
\newcommand{\Ez}{\textbf{E}_z}
\newcommand{\eQ}{\textbf{e}_Q}
\newcommand{\ex}{\textbf{e}_x}
\newcommand{\ey}{\textbf{e}_y}
\newcommand{\ez}{\textbf{e}_z}
\newcommand{\exprime}{\textbf{e}_x^{\prime}}
\newcommand{\eyprime}{\textbf{e}_y^{\prime}}
\newcommand{\ezprime}{\textbf{e}_z^{\prime}}
\newcommand{\exdprime}{\textbf{e}_x^{\prime\prime}}
\newcommand{\eydprime}{\textbf{e}_y^{\prime\prime}}
\newcommand{\ezdprime}{\textbf{e}_z^{\prime\prime}}
\newcommand{\pone}{\textbf{p}_1}
\newcommand{\ptwo}{\textbf{p}_2}
\newcommand{\pthree}{\textbf{p}_3}
\newcommand{\qone}{\textbf{q}_1}
\newcommand{\qtwo}{\textbf{q}_2}
\newcommand{\qthree}{\textbf{q}_3}
\newcommand{\eone}{\textbf{e}_1}
\newcommand{\etwo}{\textbf{e}_2}
\newcommand{\ethree}{\textbf{e}_3}
\newcommand{\eoneprime}{\eone^{\prime}}
\newcommand{\etwoprime}{\etwo^{\prime}}
\newcommand{\ethreeprime}{\ethree^{\prime}}
\newcommand{\eonedprime}{\eone^{\prime\prime}}
\newcommand{\etwodprime}{\etwo^{\prime\prime}}
\newcommand{\ethreedprime}{\ethree^{\prime\prime}}
\newcommand{\uone}{\textbf{u}_1}
\newcommand{\utwo}{\textbf{u}_2}
\newcommand{\uthree}{\textbf{u}_3}
\newcommand{\Eone}{\textbf{E}_1}
\newcommand{\Etwo}{\textbf{E}_2}
\newcommand{\Ethree}{\textbf{E}_3}
\newcommand{\ux}{\textbf{u}_x}
\newcommand{\uy}{\textbf{u}_y}
\newcommand{\ur}{\textbf{u}_r}
\newcommand{\uth}{\textbf{u}_{\theta}}
\newcommand{\uz}{\textbf{u}_z}
\newcommand{\bfet}{\textbf{e}_t}
\newcommand{\bfeb}{\textbf{e}_b}
\newcommand{\bfen}{\textbf{e}_n}
\newcommand{\dbfet}{\dot{\textbf{e}}_t}
\newcommand{\dfeb}{\dot{\textbf{e}}_b}
\newcommand{\dbfen}{\dot{\textbf{e}}_n}
\newcommand{\eteneb}{\left\{\bfet,\bfen,\bfeb \right\}}
\newcommand{\er}{\textbf{e}_r}
\renewcommand{\eth}{\textbf{e}_{\theta}}
\newcommand{\ephi}{\textbf{e}_{\phi}}
\newcommand{\dg}{\dot{g}}
\newcommand{\dq}{\dot{q}}
\newcommand{\ds}{\dot{s}}
\newcommand{\dx}{\dot{x}}
\newcommand{\dy}{\dot{y}}
\newcommand{\dz}{\dot{z}}
\newcommand{\dbfx}{\dot{\mybold{x}}}
\newcommand{\dbfy}{\dot{\mybold{y}}}
\newcommand{\dbfX}{\dot{\mybold{X}}}
\newcommand{\dbfY}{\dot{\mybold{Y}}}
\newcommand{\dex}{\dot{\textbf{e}}_x}
\newcommand{\dey}{\dot{\textbf{e}}_y}
\newcommand{\dez}{\dot{\textbf{e}}_z}
\newcommand{\deone}{\dot{\textbf{e}}_1}
\newcommand{\detwo}{\dot{\textbf{e}}_2}
\newcommand{\dethree}{\dot{\textbf{e}}_3}
\newcommand{\der}{\dot{\textbf{e}}_r}
\newcommand{\deth}{\dot{\textbf{e}}_{\theta}}
\newcommand{\dEz}{\dot{\textbf{E}}_{z}}
\newcommand{\ddg}{\ddot{g}}
\newcommand{\ddq}{\ddot{q}}
\newcommand{\dr}{\dot{r}}
\newcommand{\dss}{\ddot{s}}
\newcommand{\ddx}{\ddot{x}}
\newcommand{\ddy}{\ddot{y}}
\newcommand{\ddz}{\ddot{z}}
\newcommand{\ddr}{\ddot{r}}
\newcommand{\ddth}{\ddot{\theta}}
\newcommand{\ddbfX}{\ddot{\mybold{X}}}
\newcommand{\ddbfY}{\ddot{\mybold{Y}}}
\newcommand{\Aprime}{A^{\prime}}
\newcommand{\aprime}{a^{\prime}}
\newcommand{\bfb}{\mybold{b}}
\newcommand{\bfbprime}{\mybold{b}^{\prime}}
\newcommand{\bprime}{b^{\prime}}
\newcommand{\bfe}{\textbf{e}}
\newcommand{\bfeprime}{\textbf{e}^{\prime}}
\newcommand{\bfE}{\textbf{E}}
\newcommand{\bfEprime}{\textbf{E}^{\prime}}
\newcommand{\bfEdprime}{\textbf{E}^{\prime\prime}}
\newcommand{\bff}{\mybold{f}}
\newcommand{\bfi}{\mybold{i}}
\newcommand{\bfrho}{\mygreekbold{\rho}}
\newcommand{\rhoprime}{\rho^{\prime}}
\newcommand{\bfp}{\mybold{p}}
\newcommand{\bfq}{\mybold{q}}
\newcommand{\dbfq}{\dot{\mybold{q}}}
\newcommand{\ddbfq}{\ddot{\mybold{q}}}
\newcommand{\bfpbar}{\bar{\mybold{p}}}
\newcommand{\Cprime}{C^{\prime}}
\newcommand{\Pprime}{P^{\prime}}
\newcommand{\Qprime}{Q^{\prime}}
\newcommand{\bfpprime}{\mybold{p}^{\prime}}
\newcommand{\bfr}{\mybold{r}}
\newcommand{\dbfr}{\dot{\bfr}}
\newcommand{\ddbfr}{\ddot{\bfr}}
\newcommand{\bfrbar}{\bar{\mybold{r}}}
\newcommand{\bfu}{\mybold{u}}
\newcommand{\bfupb}{\textbf{b}}
\newcommand{\bfupe}{\textbf{e}}
\newcommand{\bfupf}{\textbf{f}}
\newcommand{\bfupg}{\textbf{g}}
\newcommand{\bfupk}{\textbf{k}}
\newcommand{\bfupkstar}{\textbf{k}^{\ast}}
\newcommand{\bfupkone}{\textbf{k}_1}
\newcommand{\bfupktwo}{\textbf{k}_2}
\newcommand{\bfupkthree}{\textbf{k}_3}
\newcommand{\bfupkonestar}{\textbf{k}^{\ast}_1}
\newcommand{\bfupktwostar}{\textbf{k}^{\ast}_2}
\newcommand{\bfupkthreestar}{\textbf{k}^{\ast}_3}
\newcommand{\bfupx}{\textbf{x}}
\newcommand{\bfupu}{\textbf{u}}
\newcommand{\bfupw}{\textbf{w}}
\newcommand{\bfv}{\mybold{v}}
\newcommand{\dbfv}{\dot{\bfv}}
\newcommand{\bfw}{\mybold{w}}
\newcommand{\bfvprime}{\mybold{v}^{\prime}}
\newcommand{\bfvbar}{\bar{\mybold{v}}}
\newcommand{\bfvbarprime}{\bar{\mybold{v}}^{\prime}}
\newcommand{\bfvhat}{\hat{\bfv}}
\newcommand{\bfvtilde}{\tilde{\bfv}}
\newcommand{\bfphi}{\mybold{\phi}}
\newcommand{\bfs}{\mybold{s}}
\newcommand{\bft}{\mybold{t}}
\newcommand{\bfx}{\mybold{x}}
\newcommand{\bfy}{\mybold{y}}
\newcommand{\bfU}{\mybold{U}}
\newcommand{\bfa}{\mybold{a}}
\newcommand{\bfc}{\mybold{c}}
\newcommand{\bfabar}{\bar{\mybold{a}}}
\newcommand{\bfg}{\mybold{g}}
\newcommand{\bfh}{\mybold{h}}
\newcommand{\bbC}{\mathbb{C}}
\newcommand{\bbE}{\mathbb{E}}
\newcommand{\bbI}{\mathbb{I}}
\newcommand{\bbO}{\mathbb{O}}
\newcommand{\bbR}{\mathbb{R}}
\newcommand{\bbS}{\mathbb{S}}
\newcommand{\bbT}{\mathbb{T}}
\newcommand{\bbU}{\mathbb{U}}
\newcommand{\bfX}{\mybold{X}}
\newcommand{\bfY}{\mybold{Y}}
\newcommand{\bfupA}{\textbf{A}}
\newcommand{\bfupB}{\textbf{B}}
\newcommand{\bfupC}{\textbf{C}}
\newcommand{\bfupD}{\textbf{D}}
\newcommand{\bfupE}{\textbf{E}}
\newcommand{\bfupEprime}{\textbf{E}^{\prime}}
\newcommand{\bfupF}{\textbf{F}}
\newcommand{\bfupG}{\textbf{G}}
\newcommand{\bfupI}{\textbf{I}}
\newcommand{\bfupK}{\textbf{K}}
\newcommand{\bfupKstar}{\textbf{K}^{\ast}}
\newcommand{\bfupO}{\textbf{O}}
\newcommand{\bfupP}{\textbf{P}}
\newcommand{\bfupQ}{\textbf{Q}}
\newcommand{\bfupS}{\textbf{S}}
\newcommand{\bfupT}{\textbf{T}}
\newcommand{\bfupU}{\textbf{U}}
\newcommand{\bfupV}{\textbf{V}}
\newcommand{\bfupW}{\textbf{W}}
\newcommand{\bfupZ}{\textbf{Z}}
\newcommand{\bfupone}{\textbf{1}}
\newcommand{\bfA}{\mybold{A}}
\newcommand{\bfB}{\mybold{B}}
\newcommand{\bfC}{\mybold{C}}
\newcommand{\bfD}{\mybold{D}}
\newcommand{\bfd}{\mybold{d}}
\newcommand{\bfDelta}{\mygreekbold{\Delta}}
\newcommand{\bfChat}{\hat{\mybold{C}}}
\newcommand{\bfF}{\mybold{F}}
\newcommand{\bfG}{\mybold{G}}
\newcommand{\bfGdot}{\dot{\mybold{G}}}
\newcommand{\bfGprime}{\mybold{G}^{\prime}}
\newcommand{\bfGbar}{\bar{\mybold{G}}}
\newcommand{\bfGtilde}{\tilde{\mybold{G}}}
\newcommand{\bfFhat}{\hat{\mybold{F}}}
\newcommand{\bfH}{\mybold{H}}
\newcommand{\bfHbar}{\bar{\mybold{H}}}
\newcommand{\bfHprime}{\mybold{H}^{\prime}}
\newcommand{\bfHbarprime}{\bar{\mybold{H}}^{\prime}}
\newcommand{\bfHhat}{\hat{\mybold{H}}}
\newcommand{\bfHbarhat}{\hat{\bar{\mybold{H}}}}
\newcommand{\bfHdot}{\dot{\mybold{H}}}
\newcommand{\bfHbardot}{\dot{\bar{\mybold{H}}}}
\newcommand{\bfI}{\mybold{I}}
\newcommand{\bfK}{\mybold{K}}
\newcommand{\bfLambda}{\boldsymbol{\Lambda}}
\newcommand{\bfGamma}{\boldsymbol{\Gamma}}
\newcommand{\bfM}{\mybold{M}}
\newcommand{\bfMbar}{\bar{\mybold{M}}}
\newcommand{\bfMhat}{\hat{\mybold{M}}}
\newcommand{\bfMbarhat}{\hat{\bar{\mybold{M}}}}
\newcommand{\bfN}{\mybold{N}}
\newcommand{\bfNA}{\mybold{N}_A}
\newcommand{\bfNB}{\mybold{N}_B}
\newcommand{\bfNx}{\mybold{N}_x}
\newcommand{\bfNy}{\mybold{N}_y}
\newcommand{\bfNz}{\mybold{N}_z}
\newcommand{\bfP}{\mybold{P}}
\newcommand{\bfPhat}{\hat{\mybold{P}}}
\newcommand{\bfQ}{\mybold{Q}}
\newcommand{\bfR}{\mybold{R}}
\newcommand{\bfRhat}{\hat{\mybold{R}}}
\newcommand{\bfT}{\mybold{T}}
\newcommand{\bfS}{\mybold{S}}
\newcommand{\bfThat}{\hat{\mybold{T}}}
\newcommand{\bfW}{\mybold{W}}
\newcommand{\bftau}{\mygreekbold{\tau}}
% \newcommand{\bftone}{\mybold{t}_1}
% \newcommand{\bfttwo}{\mybold{t}_2}
\newcommand{\bftone}{\textbf{u}}
\newcommand{\bfttwo}{\textbf{w}}
\newcommand{\bfn}{\textbf{n}}
\newcommand{\dds}[1]{\frac{d{#1}}{ds}}
\newcommand{\ddt}[1]{\frac{d{#1}}{dt}}
\newcommand{\ddtt}[1]{\frac{d^2{#1}}{dt^2}}
\newcommand{\delt}[1]{\frac{\partial{#1}}{\partial t}}
\newcommand{\drdt}{\ddt{\bfr}}
\newcommand{\eonetwothree}{\{\eone,\etwo,\ethree\}}
\newcommand{\Eonetwothree}{\{\Eone,\Etwo,\Ethree\}}
\newcommand{\ponetwothree}{\{\pone,\ptwo,\pthree\}}
\newcommand{\konetwothree}{\{\bfupkone,\bfupktwo,\bfupkthree\}}
\newcommand{\konetwothreestar}{\{\bfupkonestar,\bfupktwostar,\bfupkthreestar\}}
\newcommand{\qonetwothree}{\{\qone,\qtwo,\qthree\}}
\newcommand{\uonetwothree}{\{\uone,\utwo,\uthree\}}
\newcommand{\exyzprime}{\{\exprime,\eyprime,\ezprime\}}
\newcommand{\eonetwothreeprime}{\{\eoneprime,\etwoprime,\ethreeprime\}}
\newcommand{\eonetwothreep}{\{\eone^p,\etwo^p,\ethree^p\}}
\newcommand{\eonetwothreedprime}{\{\eonedprime,\etwodprime,\ethreedprime\}}
\newcommand{\exyz}{\{\ex,\ey,\ez\}}
\newcommand{\Exyz}{\{\ex,\ey,\Ez\}}
\newcommand{\EXYZ}{\{\Ex,\Ey,\Ez\}}
\newcommand{\erthz}{\{\er,\eth,\ez\}}
\newcommand{\Erthz}{\{\er,\eth,\Ez\}}
\newcommand{\erthphi}{\{\er,\eth,\ephi\}}
\newcommand{\erphith}{\{\er,\ephi,\eth\}}
\newcommand{\eQthz}{\{\eQ,\eth,\ez\}}
\newcommand{\etnb}{\{\bfet,\bfen,\bfeb\}}
\newcommand{\ttnn}{\{\bftone,\bfttwo,\bfn\}}
\newcommand{\calph}{\cos\hspace{-0.5pt}\alpha}
\newcommand{\salph}{\sin\hspace{-0.5pt}\alpha}
\newcommand{\talph}{\tan\hspace{-0.5pt}\alpha}
\newcommand{\cbeta}{\cos\hspace{-0.5pt}\beta}
\newcommand{\sbeta}{\sin\hspace{-0.5pt}\beta}
\newcommand{\tbeta}{\tan\hspace{-0.5pt}\beta}
\newcommand{\csqbeta}{\cos^2\hspace{-0.5pt}\beta}
\newcommand{\ssqbeta}{\sin^2\hspace{-0.5pt}\beta}
\newcommand{\cth}{\cos\hspace{-0.5pt}\theta\hspace{1pt}}
\newcommand{\sth}{\sin\hspace{-0.5pt}\theta\hspace{1pt}}
\newcommand{\tth}{\tan\hspace{-0.5pt}\theta\hspace{1pt}}
\newcommand{\secth}{\sec\hspace{-0.5pt}\theta\hspace{1pt}}
\newcommand{\dth}{\dot{\theta}}
\newcommand{\dalpha}{\dot{\alpha}}
\newcommand{\ddalpha}{\ddot{\alpha}}
\newcommand{\uxyz}{\{\ux,\uy,\uz\}}
\newcommand{\ssqth}{\sin^2\hspace{-0.5pt}\theta}
\newcommand{\csqth}{\cos^2\hspace{-0.5pt}\theta}
\newcommand{\secsqth}{\sec^2\hspace{-.51pt}\theta}
\newcommand{\cphi}{\cos\hspace{-0.5pt}\phi}
\newcommand{\sphi}{\sin\hspace{-0.5pt}\phi}
\newcommand{\tphi}{\tan\hspace{-0.5pt}\phi}
\newcommand{\cpsi}{\cos\hspace{-0.5pt}\psi\hspace{1pt}}
\newcommand{\spsi}{\sin\hspace{-0.5pt}\psi\hspace{1pt}}
\newcommand{\dpsi}{\dot{\psi}}
\newcommand{\ddpsi}{\ddot{\psi}}
\newcommand{\dbfbdt}{\frac{d\bfb}{dt}}
\newcommand{\ddbfbdtt}{\frac{d^2\bfb}{dt^2}}
\newcommand{\dbfrdt}{\frac{d\mybold{r}}{dt}}
\newcommand{\ddbfrdtt}{\frac{d^2\mybold{r}}{dt^2}}
\newcommand{\dbfvdt}{\frac{d\mybold{v}}{dt}}
\newcommand{\dphi}{\dot{\phi}}
\newcommand{\ddphi}{\ddot{\phi}}
\newcommand{\ssqphi}{\sin^2\hspace{-0.5pt}\phi}
\newcommand{\csqphi}{\cos^2\hspace{-0.5pt}\phi}
\newcommand{\tsqphi}{\tan^2\hspace{-0.5pt}\phi}
\newcommand{\secphi}{\sec\hspace{-0.5pt}\phi}
\newcommand{\cscphi}{\csc\hspace{-0.5pt}\phi}
\newcommand{\alp}{\mygreekbold{\alpha}}
\newcommand{\bfnu}{\mygreekbold{\nu}}
\newcommand{\bfom}{\mygreekbold{\omega}}
\newcommand{\dbfom}{\dot{\mygreekbold{\omega}}}
\newcommand{\bfomprime}{\mygreekbold{\omega}^{\prime}}
\newcommand{\omprime}{\omega^{\prime}}
\newcommand{\bfOm}{\mygreekbold{\Omega}}
\newcommand{\bfpi}{\mygreekbold{\pi}}
\newcommand{\bfpsi}{\mygreekbold{\psi}}
\newcommand{\omx}{\omega_x}
\newcommand{\omy}{\omega_y}
\newcommand{\omz}{\omega_z}
\newcommand{\omxx}{\omega_{xx}}
\newcommand{\omxy}{\omega_{xy}}
\newcommand{\omxz}{\omega_{xz}}
\newcommand{\omyx}{\omega_{yx}}
\newcommand{\omyy}{\omega_{yy}}
\newcommand{\omyz}{\omega_{yz}}
\newcommand{\omzx}{\omega_{zx}}
\newcommand{\omzy}{\omega_{zy}}
\newcommand{\omzz}{\omega_{zz}}
\newcommand{\omoneone}{\omega_{11}}
\newcommand{\omonetwo}{\omega_{12}}
\newcommand{\omonethree}{\omega_{13}}
\newcommand{\omtwoone}{\omega_{21}}
\newcommand{\omtwotwo}{\omega_{22}}
\newcommand{\omtwothree}{\omega_{23}}
\newcommand{\omthreeone}{\omega_{31}}
\newcommand{\omthreetwo}{\omega_{32}}
\newcommand{\omthreethree}{\omega_{33}}
\newcommand{\omone}{\omega_1}
\newcommand{\omtwo}{\omega_2}
\newcommand{\omthree}{\omega_3}
\newcommand{\lsup}[2]{{\vphantom{#2}}^{#1}{{#2}}}
\newcommand{\lrsup}[3]{{\vphantom{#2}}^{#1}{{#2}}^{#3}}
\newcommand{\lsupddt}[2]{{\vphantom{\frac{d#2}{dt}}}^{#1}{\frac{d#2}{dt}}}
\newcommand{\lsupddtau}[2]{{\vphantom{\frac{d#2}{d\tau}}}^{#1}{\frac{d#2}{d\tau}}}
\newcommand{\lsupdds}[2]{{\vphantom{\frac{d#2}{ds}}}^{#1}{\frac{d#2}{ds}}}
\newcommand{\lrsupddt}[3]{{\vphantom{\frac{d#2}{dt}}}^{#1}{\frac{d#2}{dt}}^{#3}}
\newcommand{\lsupdddtt}[2]{{\vphantom{\frac{d^2#2}{dt^2}}}^{#1}{\frac{d^2#2}{dt^2}}}
\newcommand{\lsupdot}[2]{{\vphantom{\dot{#2}}}^{#1}{\dot{#2}}}
\newcommand{\lsupddot}[2]{{\vphantom{\ddot{#2}}}^{#1}{\ddot{#2}}}
% \newcommand{\dotrel}[1]{\left(\frac{\partial #1}{\partial t}\right)}
% \newcommand{\ddotrel}[1]{\left(\frac{\partial^2{#1}}{\partial t^2}\right)}
\newcommand{\dotrel}[1]{\left(\frac{d #1}{dt}\right)}
\newcommand{\ddotrel}[1]{\left(\frac{d^2{#1}}{dt^2}\right)}
\newcommand{\vdotrel}{\left(\frac{\partial \bfv}{\partial t}\right)}
\newcommand{\rdotcrel}{\left(\frac{\partial \bfr_C}{\partial t}\right)}
\newcommand{\vdotcrel}{\left(\frac{\partial \bfv_C}{\partial t}\right)}
\newcommand{\rdotprel}{\left(\frac{\partial \bfr_P}{\partial t}\right)}
\newcommand{\vdotprel}{\left(\frac{\partial \bfv_P}{\partial t}\right)}
\newcommand{\rhodotrel}{\left(\frac{\partial \bfrho}{\partial t}\right)}
\newcommand{\rhoddotrel}{\left(\frac{\partial^2 \bfrho}{\partial t^2}\right)}
\newcommand{\omdotrel}{\left(\frac{\partial \bfom}{\partial t}\right)}
\newcommand{\omegacrossr}{\mygreekbold{\omega}\times\bfr}
\newcommand{\Omegacrossr}{\mygreekbold{\Omega}\times\bfr}
\newcommand{\omegacrossv}{\mygreekbold{\omega}\times\bfv}
\newcommand{\Omegacrossv}{\mygreekbold{\Omega}\times\bfv}
\newcommand{\bfalpha}{\mygreekbold{\alpha}}
\newcommand{\bfbeta}{\mygreekbold{\beta}}
\newcommand{\bfTheta}{\mygreekbold{\Theta}}
\newcommand{\speed}{\|\bfv\|}
\newcommand{\bfzero}{\mybold{0}}
\newcommand{\vhat}{\hat{v}}
\newcommand{\vtilde}{\tilde{v}}
\newcommand{\vprime}{v^{\prime}}
\newcommand{\vbar}{\bar{v}}
\newcommand{\vbarprime}{\bar{v}^{\prime}}
\newcommand{\Chat}{\hat{C}}
\newcommand{\Fhat}{\hat{F}}
\newcommand{\Phat}{\hat{P}}
\newcommand{\Rhat}{\hat{R}}
\newcommand{\That}{\hat{T}}
\newcommand{\dUdr}{\frac{\partial U}{\partial \bfr}}
\newcommand{\mcc}{\mathcal{c}}
\newcommand{\mcg}{\mathcal{g}}
\newcommand{\mcs}{\mathcal{s}}
\newcommand{\mcA}{\mathcal{A}}
\newcommand{\mcB}{\mathcal{B}}
\newcommand{\mcC}{\mathcal{C}}
\newcommand{\mcD}{\mathcal{D}}
\newcommand{\mcE}{\mathcal{E}}
\newcommand{\mcF}{\mathcal{F}}
\newcommand{\mcG}{\mathcal{G}}
\newcommand{\mcK}{\mathcal{K}}
\newcommand{\mcL}{\mathcal{L}}
\newcommand{\mcM}{\mathcal{M}}
\newcommand{\mcN}{\mathcal{N}}
\newcommand{\mcP}{\mathcal{P}}
\newcommand{\mcQ}{\mathcal{Q}}
\newcommand{\mcR}{\mathcal{R}}
\newcommand{\mcT}{\mathcal{T}}
\newcommand{\mcU}{\mathcal{U}}
\newcommand{\mcV}{\mathcal{V}}
\newcommand{\bfmcK}{\mygreekbold{\mathcal{K}}}
\newcommand{\mcS}{\mathcal{S}}
\newcommand{\mcX}{\mathcal{X}}
\newcommand{\mcY}{\mathcal{Y}}
\newcommand{\dmcg}{\dot{\mathcal{g}}}
\newcommand{\ddmcg}{\ddot{\mathcal{g}}}
\newcommand{\dmcs}{\dot{\mathcal{s}}}
\newcommand{\ddmcs}{\ddot{\mathcal{s}}}
\newcommand{\dbfR}{\dot{\bfR}}
\newcommand{\dbfrho}{\dot{\bfrho}}
\newcommand{\ddbfR}{\ddot{\bfR}}
\newcommand{\ddbfrho}{\ddot{\bfrho}}
\newcommand{\eg}{e.g.,~}
\newcommand{\ie}{i.e.,~}
\newcommand{\bbone}{\mathbb{1}}
\newcommand{\pd}[2]{\frac{\partial #1}{\partial #2}}
\newcommand{\pdrf}[3]{\frac{\partial #2}{\hspace{-4pt}\lsup{#1}{\partial #3}}}
\newcommand{\tr}{^{\sf T}}
\newcommand{\bfone}{\mybold{1}}



\maketitle
\clearpage


\section*{Preface}

$\mathbb{GPOPS-II}$ is a general-pupose software for solving nonlinear
optimal control problems that arise in a wide variety applications
including engineering, economics, and medicine.  $\mathbb{GPOPS-II}$ uses
some of the latest advancements in the area of pseudospectral methods
for solving optimal control problems.  $\mathbb{GPOPS-II}$ employs an 
$hp$-adaptive Radau pseudospectral Gaussian quadrature method where
the collocation is performed at the Legendre-Gauss-Radau quadrature
points.  $\mathbb{GPOPS-II}$ has been designed to work with the nonlinear
programming (NLP) solvers SNOPT and IPOPT, and MATLAB mex files for
both SNOPT and IPOPT are included with the software.  $\mathbb{GPOPS-II}$
employs sparse finite-differencing to estimate all first and second
derivatives required by the NLP solver.  The software has been
designed to be extremely flexible, allowing a user to formulate an
optimal control problem in a way that makes sense for the problem
being solved.  Few, if any, restrictions have been placed on the
manner in which a problem needs to be modeled.  As stated, the
software is {\em general-purpose}, that is, it has not been developed
for any specific type of problem.  While the developers of
$\mathbb{GPOPS-II}$ make no guarantee as to the fitness of the software for
any particular purpose, it is certainly hoped that software is useful
for a variety of applications.

\section*{Complete Overhaul from Previous Versions of GPOPS}

$\mathbb{GPOPS-II}$ represents a complete overhaul from the GPOPS software that was
released between 2008 and 2012.  Specifically, this new software,
$\mathbb{GPOPS-II}$, is organized in a completely different manner from GPOPS and
has significantly more functionality from GPOPS.  Furthermore, $\mathbb{GPOPS-II}$
does {\em not} maintain backward compatibility with GPOPS.  While the
authors of $\mathbb{GPOPS-II}$ realize that the lack of backward compatibility may
be inconvenient for some users, the increased power and functionality
of $\mathbb{GPOPS-II}$ will make it worth the short term inconvenience of the
transition.  In order to have as smooth a transition as possible to
the new software, the authors of $\mathbb{GPOPS-II}$ are happy to assist users of
GPOPS in rewriting their code for $\mathbb{GPOPS-II}$.  

\section*{Acknowledgments}

The authors gratefully acknowledge support for this research from the 
U.S.~Office of Naval Research (ONR) under Grant N00014-11-1-0068 and 
from the U.S.~Defense Advanced Research Projects Agency (DARPA) Under 
Contract HR0011-12-0011.

\section*{Disclaimer}

The views expressed are those of the authors and do not reflect the
official policy or position of the Department of Defense or the
U.S.~Government.   Furthermore, the contents of this document and the 
corresponding software are provided ``as is'' without any
merchantability or fitness for any particular application.  Neither
authors nor their employers (past, present, or future) assume any
responsibility whatsoever from any harm resulting from the
software. The authors do, however, hope that users will find this
software useful for research and other purposes.   

\section*{Licensing Agreement}

By downloading, using, modifying, or distributing $\mathbb{GPOPS}2$, you agree to 
the terms of this license agreement.  This license gives you extremely
GENEROUS RIGHTS, so if you do not agree to the terms of this
agreement, you may not proceed further with using, using, modifying,
or distributing $\mathbb{GPOPS}2$.  

\subsection*{License for $\mathbb{GPOPS}2$ Software}

This is a license for the general-purpose optimal control software
$\mathbb{GPOPS}2$.  The license for $\mathbb{GPOPS}2$ is based on the Simple Public
License.  In the same spirit as the Simple Public 
License, the language for the $\mathbb{GPOPS}2$ License is similar to that
of GPL 2.0.  The words are different, but the goal is the same:  
to guarantee for all users the freedom to share and change       
software.  If anyone wonders about the meaning of the $\mathbb{GPOPS}2$ 
License, they should interpret it as consistent with GPL 2.0.    
                                                                 
The $\mathbb{GPOPS}2$ License applies to the software's source and object    
code and comes with any rights that I have in it (other than     
trademarks). You agree to the $\mathbb{GPOPS}2$ License simply by            
downloading, copying, distributing, or making a derivative work  
of the software.  You get the royalty-free right to
\begin{itemize}
\item Use the software for any purpose;
\item Make derivative works of it (this is called a "Derived Work");                                            
\item Copy and distribute it and any Derived Work.
\end{itemize}
If you distribute the software or a Derived Work, you must give back
to the community by
\begin{itemize} 
\item Prominently noting the date of any changes you make;       
\item Leaving other people's copyright notices, warranty disclaimers,
and license terms in place;
\item Providing the source code, build scripts, installation scripts,
and interface definitions in a form that is easy to get and best to
modify;
\item Licensing it to everyone under the terms of this license
agreement without adding further restrictions to the rights provided;
\item Conspicuously announcing that it is available under this
license.
\end{itemize}

\subsection*{Distribution of $\mathbb{GPOPS}2$}

$\mathbb{GPOPS}2$ is a completely free software both for use and for
redistribution.  Furthermore, while it may be used within commercial
organizations, it is not for sale or resale.  The only exception to
the sales restriction above is that $\mathbb{GPOPS}2$ may be included as a part of
a free open-source software (for example, a distribution of the Linux
operating system).  When distributing $\mathbb{GPOPS}2$ with a free
operating system, no fee beyond the price of the operating system
itself may be added (that is, you cannot profit from the
redistribution of $\mathbb{GPOPS}2$). $\mathbb{GPOPS}2$ is not for commercial use with
the exception that it may be used by commercial organizations for
internal research and development.  Any use of $\mathbb{GPOPS}2$ by commercial
organizations that involve the presentation of results for
profit-making purposes is strictly prohibited.  In addition, there are
some things that you must shoulder:
\begin{itemize}
\item You get {\em no warranties} of any kind;
\item If the software damages you in any way, you may only recover
direct damages up to the amount you paid for it (that is, you get zero
if you did not pay anything for the software);
\item You may not recover any other damages, including those called
"consequential damages." (The state or country where you live may not
allow you to limit your liability in this way, so this may not apply
to you).
\end{itemize}
The $\mathbb{GPOPS}2$ License continues perpetually, except that your license
rights end automatically if
\begin{itemize}
\item You do not abide by the "give back to the community" terms (your
licensees get to keep their rights if they abide);
\item Anyone prevents you from distributing the software under the
terms of this license agreement.
\item You sell the software in any manner with the one exception
listed above.
\end{itemize}

{\em $\mathbb{GPOPS}2$ is provided ``as is'' without warranty
of any kind, expressed or implies, including but not limited to the 
warranties of merchantability, fitness for a particular purpose, and
non-infringement.  In no event shall the authors or copyright holders
be liable for any claim, damages, or other liability, whether in an
action of contract, tort, or otherwise, arising from, out of, or in
connection with the software or the use or dealings in the software}.


\clearpage
\setcounter{tocdepth}{2}
\tableofcontents

\clearpage

\section{Introduction to the General-Purpose Software $\mathbb{GPOPS-II}$}

A $P$-phase optimal control problem can be stated in the following
general form.  Determine the state, $\bfy^{(p)}(t)\in\mathbb{R}^{n_y^{(p)}}$, control,
$\bfu^{(p)}(t)\in\mathbb{R}^{n_u^{(p)}}$, initial time, $t_0^{(p)}\in\mathbb{R}$, final time, $t_f^{(p)}\in\mathbb{R}$,
integrals, $\bfd^{(p)}\in\mathbb{R}^{n_d^{(p)}}$, in each phase $p\in[1,\ldots,P]$, and the
static parameters, $\bfs\in\mathbb{R}^{n_s}$, that minimize the cost functional  
\begin{equation}
  J = \phi\left[\bfy^{(1)}(t_0^{(1)}),\ldots,\bfy^{(P)}(t_0^{(P)}),t_0^{(1)},\ldots,t_0^{(P)},\bfy^{(1)}(t_f^{(1)}),\ldots,\bfy^{(P)}(t_f^{(P)}),t_f^{(1)},\ldots,t_f^{(P)},\bfq^{(1)},\ldots,\bfq^{(P)},\bfs\right]
\end{equation}
subject to the dynamic constraints
\begin{equation}
  \dbfy^{(p)} = \bfa^{(p)}\left[\bfy^{(p)},\bfu^{(p)},t^{(p)},\bfs\right], \qquad (p=1,\ldots,P),
\end{equation}
the event constraints
\begin{equation}
  \small
  \begin{array}{c}
  \bfb_{\min}^{(g)} \leq
\bfb\left[\bfy^{(1)}(t_0^{(1)}),\ldots,\bfy^{(P)}(t_0^{(P)}),t_0^{(1)},\ldots,t_0^{(P)},\bfy^{(1)}(t_f^{(1)}),\ldots,\bfy^{(P)}(t_f^{(P)}),t_f^{(1)},\ldots,t_f^{(P)},\bfq^{(1)},\ldots,\bfq^{(P)},\bfs\right]
  \leq \bfb_{\max}^{(g)}, \\ (g=1,\ldots,G), \end{array}
\end{equation}
the inequality path constraints
\begin{equation}
  \bfc_{\min}^{(p)} \leq
  \bfc^{(p)}\left[\bfy^{(p)},\bfu^{(p)},t^{(p)},\bfs\right]\leq \bfc_{\max}^{(p)}, \qquad   (p=1,\ldots,P),
\end{equation}
and the integral constraints
\begin{equation}
  \bfq_{\min}^{(p)} \leq \bfq^{(p)} \leq \bfq_{\max}^{(p)}, \quad (p=1,\ldots,P)
\end{equation}
where
\begin{equation}
  q_i^{(p)} = \int_{t_0^{(p)}}^{t_f^{(p)}} Q_i\left[\bfy^{(p)}(t),\bfu^{(p)}(t),t,\bfs\right]dt,\quad (i=1,\ldots,n_q^{(p)};p=1,\ldots,P).
\end{equation}
While much of the time a user may want to solve a problem consisting of
multiple phases, it is important to note that the phases {\em need not
  be sequential}.  To the contrary, any two phases may be linked 
provided that the independent variable does not change direction (\ie the
independent variable moves in the same direction during each phase that is
linked).  

\subsection{Radau Pseudospectral Method Employed by $\mathbb{GPOPS-II}$}

The method employed by $\mathbb{GPOPS-II}$ is an $hp$-adaptive version of
the {\em Radau pseudospectral method}.  The Radau pseudospectral
method is an orthogonal collocation Gaussian quadrature implicit
integration method where  collocation is performed at the {\em
  Legendre-Gauss-Radau} points. The theory behind the Radau
pseudospectral method used in $\mathbb{GPOPS-II}$ can be found in 
Refs.~\citen{Garg1}, \citen{Garg2}, \citen{Garg3}, and
\citen{Patterson2011}.   

\subsection{Organization of $\mathbb{GPOPS-II}$}

$\mathbb{GPOPS-II}$ is organized as follows.  In order to specify the
optimal control problem that is to be solved, the user must write
the following MATLAB functions: (1) an endpoint function; and (2) a
continuous function.  The endpoint function defines how the the start and/or
terminus in any of the phases in the problem, the integrals in any
phase of the problem and the static parameters are related to one
another.  The endpoint function also defines the cost to be minimized.
The {\em continuous} function defines the evolution of the dynamics in
any phase of the problem, the integrands that are required to compute
any integrals in any phase of the problem, and any path constraints in
any phase of the problem.  Next, the user must specify the lower and
upper limits on the following quantities: 
\begin{enumerate}[(1)]
  \item the time at the start and terminus of a phase; 
  \item the state at the start of a phase, during a phase, and at the
    terminus of a phase;
 \item the control during a phase;
  \item the path constraints
 \item the event constraints;
  \item the static parameters.  
\end{enumerate}
The remainder of this document is devoted to describing in detail the
MATLAB syntax for describing the optimal control problem and each of
the constituent functions. 

\subsection{Color Highlighting Throughout Document}

The following notation is adopted for use throughout the remainder of
this document.  First, all user-specified names will be denoted by \slred{red
  slanted} characters.  Second, any item denoted by \bfblue{blue
  boldface} characters  are {\underline pre-defined} and cannot be changed by
the user.  Users who do not have color rendering capability will see
only slanted and boldface characters, respectively.  


\section{Constructing an Optimal Control Problem Using $\mathbb{GPOPS-II}$}

We now proceed to describe the constructs required to specify an
optimal control problem in $\mathbb{GPOPS-II}$.  We note that the key
MATLAB programming elements used in constructing an optimal control
problem in $\mathbb{GPOPS-II}$ are {\em structure} and {\em arrays of structures}.
In this Section we provide the details of constructing a problem using
$\mathbb{GPOPS-II}$.  First, the call to $\mathbb{GPOPS-II}$ s given as
\begin{center}
 \slred{output}=\bfblue{gpops2}(\slred{input}),
\end{center} 
where \slred{input} is a user-defined structure that contains all
of the information about the optimal control problem to be solved and
\slred{output} is a structure that contains the information obtained
by solving the optimal control problem.  In this section we describe
the contents of the structures \slred{input} and \slred{output}.  

\subsection{Syntax for Input Structure \bfblue{setup}  \label{sect:input-syntax}} 

The user-defined structure \slred{setup} contains required fields and
optional fields.  The required fields in the structure \slred{input}
are as follows: 
\begin{itemize}
\item \bfblue{name}:  a string {\em with no blank spaces} that
  contains the name of the problem;  
\item \bfblue{functions}:  a structure that contains the name of the
  continuous function and the endpoint function (see Section
  \ref{sect:functions} for further details); 
\item \bfblue{bounds}:  an structure that contains the information
  about the lower and upper bounds on the different variables and 
  constraints in the problem (see Section \ref{sect:bounds} for further
  details); 
\item \bfblue{guess}:  an structure that contains a guess of the
  time, state, control, integrals, and static parameters in the
  problem (see Section \ref{sect:guess} for further details); 
\end{itemize}
In addition to the above required fields in the structure
\slred{setup}, {\em optional} fields in the \slred{setup} structure
may be specified (these fields may be provided by the user if it may
of benefit for a particular problem of interest). The optional fields
in the structure \slred{setup} are given as follows along with the
list of possible values and the default values:
\begin{itemize}
\item \bfblue{auxdata}:  a structure containing auxiliary data that
  may be used by different functions in the problem.  Including
  \bfblue{auxdata} eliminates any need to specify global variables for
  use in the problem.  The following table provided the possible
 values and their defaults for the field \slred{setup}.\bfblue{auxdata}:
\begin{center}
  \begin{tabular}{|c|c|c|} \hline
  {\bf Field} & {\bf Possible Values} & {\bf Default} \\\hline
  \slred{setup}.\bfblue{auxdata} & Any Problem-Specific Data & Not
  Provided \\ \hline
\end{tabular}
\end{center}
\item \bfblue{derivatives}:   a structure that specifies the
  derivative approximation to be used by the NLP solver and the
  derivative order ('first' or 'second') to be used by the NLP
  solver.  The field \slred{setup}.\bfblue{derivatives} contains three
  fields \bfblue{supplier}, \bfblue{derivativelevel}, and \bfblue{dependencies} where the field  
  \slred{setup}.\bfblue{derivatives}.\bfblue{supplier} contains the
  type of derivative approximation , the field 
  \slred{setup}.\bfblue{derivatives}.\bfblue{derivativelevel} contains
  the derivative order, while the field \slred{setup}.\bfblue{derivatives}.\bfblue{dependencies}
  determines how the dependencies are found.
   The following table provided the possible
 values and their defaults for the field \slred{setup}.\bfblue{derivatives}:
  \begin{center}
 \begin{tabular}{|c|c|c|} \hline
 {\bf Field} & {\bf Possible Values} & {\bf Default} \\\hline
 \slred{setup}.\bfblue{derivatives}.\bfblue{supplier} & 'sparseFD', 'sparseBD' or 'sparseCD' & 'sparseFD' \\ \hline
 \slred{setup}.\bfblue{derivatives}.\bfblue{derivativelevel} & 'first' or 'second' & 'first' \\ \hline
 \slred{setup}.\bfblue{derivatives}.\bfblue{dependencies} & 'full', 'sparse' or 'sparseNaN' & 'sparseNaN' \\ \hline
\end{tabular}
\end{center}
\item \bfblue{scales}: a structure that specifies the type of scaling
  to be used when solving the problem.  [{\bf Possible Values:}
 'none' or 'automatic-bounds'; {\bf Default:} 'none'];
  \begin{center}
\begin{tabular}{|c|c|c|} \hline
{\bf Field} & {\bf Possible Values} & {\bf Default} \\\hline
\slred{setup}.\bfblue{scales} & 'none' or 'automatic-bounds' & 'none' \\ \hline
\end{tabular}
\end{center}
\item \bfblue{mesh}: a structure that specifies the
 information as to the type of mesh refinement method to be used and  
  the mesh refinement accuracy tolerance, as well as the initial mesh.  The structure
  \slred{setup}.\bfblue{mesh} contains the fields
  \bfblue{method}, \bfblue{tolerance}, \bfblue{maxiteration}, and \bfblue{phase}.  The field
  \slred{setup}.\bfblue{mesh}.\bfblue{method} is a string that
  specified the particular mesh refinement method to be used, while
  the field \slred{setup}.\bfblue{mesh}.\bfblue{tolerance} contains the
  desired accuracy tolerance of the mesh, while the field \slred{setup}.\bfblue{mesh}.\bfblue{maxiterations}
  contains the maximum number of allowed mes iterations.
  \begin{center}
   \begin{tabular}{|c|c|c|} \hline
   {\bf Field} & {\bf Possible Values} & {\bf Default} \\\hline
   \slred{setup}.\bfblue{mesh}.\bfblue{method} & 'hp' or 'hp1' & 'hp1' \\\hline
  \slred{setup}.\bfblue{mesh}.\bfblue{options}.\bfblue{tolerance} &
 Positive Number Between $0$ and $1$ & $10^{-3}$ \\\hline
  \slred{setup}.\bfblue{mesh}.\bfblue{options}.\bfblue{maxiteration} &
  Non-Negative Integer & 10 \\ \hline
\end{tabular}
 \end{center}
  The field \bfblue{mesh}.\bfblue{phase} specifies the initial mesh intervals 
  in a given phase and the number of collocation (Radau) points in
  each mesh interval.  The field \slred{setup}.\bfblue{mesh}.\bfblue{phase($p$)}.\bfblue{fraction}
  contains the mesh intervals for each phase $p=1,\ldots,P$,
  where the mesh intervals are specified in a row
  vector that provides the fraction of a scaled interval $[0,1]$ that
  corresponds to each mesh interval.  The field \slred{setup}.\bfblue{mesh}.\bfblue{phase($p$)}.\bfblue{colpoints}
  contains the number of collocation points in each phase $p=1,\ldots,P$,
  where the number of collocation points in each mesh interval is also specified as a
  row vector such that the $i^{th}$ entry in \slred{setup}.\bfblue{mesh}.\bfblue{phase($p$)}.\bfblue{colpoints}
  corresponds to the $i^{th}$ entry in \slred{setup}.\bfblue{mesh}.\bfblue{phase($p$)}.\bfblue{fraction}.
  \begin{center}
\begin{tabular}{|c|c|c|} \hline
{\bf Field} & {\bf Possible Values} & {\bf Default} \\ \hline
\slred{setup}.\bfblue{mesh}.\bfblue{phase($p$)}.\bfblue{fraction} &\begin{tabular}{c} Row
  Vector of Length $M\geq 1$ of Positive \\
  Numbers $>0$ and $<1$ that Sum to Unity \end{tabular} & 0.1*ones(1,10) \\ \hline
\slred{setup}.\bfblue{mesh}.\bfblue{phase($p$)}.\bfblue{colpoints} & \begin{tabular}{c}
  Row Vector of Length $M\geq 1$ \\ of Positive Integers $>1$ and
  $<10$ \\ ($M$ is the same as in
  \slred{setup}.\bfblue{mesh}.\bfblue{percent}) \end{tabular} & 4*ones(1,10) \\ \hline
\end{tabular}
\end{center}
 
\end{itemize} 
It is important to note that $\mathbb{GPOPS-II}$ has been designed so that
the independent variable must be monotonically {\em increasing} in
each phase of the problem.  

\subsection{Syntax for Structure \slred{setup}.\bfblue{functions} \label{sect:functions}}

The syntax for specifying the names of the MATLAB functions given in
\slred{setup}.\bfblue{functions} given as follows:
\begin{displaymath}
  \begin{array}{lcl}
   \slred{setup}.\bfblue{functions.continuous} & = & \slred{@continuousfun.m} \\
   \slred{setup}.\bfblue{functions.endpoint} & = & \slred{@endpointfun.m} \\
 \end{array}
\end{displaymath}
The details of the syntax for each function are provided in Sections
\ref{sect:endpointSyntax} and \ref{sect:continuousSyntax}.  

\subsection{Syntax for \bfblue{bounds} Structure \label{sect:bounds}}

Once the user-defined structure \slred{input} has been defined, the 
next step in setting up a problem for use with $\mathbb{GPOPS-II}$ is to
create the structure \slred{input}.\bfblue{bounds}. The structure
\bfblue{bounds} contains the following {\em three} fields:
\bfblue{phase}, \bfblue{parameters}, and \bfblue{eventgroup}.  The 
field \slred{input}.\bfblue{bounds}.\bfblue{phase} is an array of structures
of length $P$ (where $P$ is the number of phases) that specifies the
bounds on the time, state, control, path constraints, and integrals in
each phase $p=1,\ldots,P$ of the problem.  The field
\slred{input}.\bfblue{bounds}.\bfblue{parameters} contains the lower
and upper bounds on the static parameters in the problem.  The field 
\slred{input}.\bfblue{bounds}.\bfblue{eventgroup} is an array of
structures of length $G$, where $G$ is the number of event groups in
the problem.  The $p^{th}$ element in the array of structures
\slred{input}.\bfblue{bounds}.\bfblue{phase} contains the following fields:
\begin{itemize}
\item \bfblue{bounds.phase($p$).initialtime.lower} and
  \bfblue{bounds.phase($p$).initialtime.upper}: scalars that contain the
  information about the lower and upper bounds on the initial time in
  phase $p\in[1,\ldots,P]$.  The scalars
  \bfblue{bounds.phase($p$).initialtime.lower} and
  \bfblue{bounds.phase($p$).initialtime.upper} have the following form: 
 \begin{displaymath}
   \begin{array}{lcl}
     \bfblue{bounds.phase($p$).initialtime.lower} & = &  t_0^{\textrm{lower}} \\
     \bfblue{bounds.phase($p$).initialtime.upper} & = &  t_0^{\textrm{upper}}
  \end{array}
 \end{displaymath}
\item \bfblue{bounds.phase($p$).finaltime.lower} and \bfblue{bounds($p$).finaltime.upper}:
 scalars that contain the information about the lower and upper
 bounds on the final time in phase $p\in[1,\ldots,P]$.  The
 scalars \bfblue{bounds.phase($p$).finaltime.lower} and \bfblue{bounds.phase($p$).finaltime.upper}
 have the following form:
 \begin{displaymath}
   \begin{array}{lcl}
     \bfblue{bounds.phase($p$).finaltime.lower} & = &  t_f^{\textrm{lower}} \\
     \bfblue{bounds.phase($p$).finaltime.upper} & = &  t_f^{\textrm{upper}}
  \end{array}
 \end{displaymath}
\item \bfblue{bounds.phase($p$).initialstate.lower} and
  \bfblue{bounds.phase($p$).initialstate.upper}: 
row vectors of length $n_y^{(p)}$ that contain the lower and upper
bounds on the initial state in phase $p\in[1,\ldots,P]$.  The row
vectors \bfblue{bounds.phase($p$).initialstate.lower} and
\bfblue{bounds.phase($p$).initialstate.upper} have the following form:
\begin{displaymath}
 \begin{array}{lcl}
   \bfblue{bounds.phase($p$).initialstate.lower} & = &
  \left[\begin{array}{ccc} y_{0,1}^{\textrm{lower}} & \cdots & y_{0,n_y^{(p)}}^{\textrm{lower}}\end{array}\right] \\
   \bfblue{bounds.phase($p$).initialstate.upper} & = &
  \left[\begin{array}{ccc} y_{0,1}^{\textrm{upper}} & \cdots & y_{0,n_y^{(p)}}^{\textrm{upper}}\end{array}\right]
\end{array}
\end{displaymath}
\item \bfblue{bounds.phase($p$).state.lower} and \bfblue{bounds.phase($p$).state.upper}:
row vectors of length $n_y^{(p)}$ that contain the lower and upper
bounds on the state during phase $p\in[1,\ldots,P]$.  The row
vectors \bfblue{bounds.phase($p$).state.lower} and
\bfblue{bounds.phase($p$).state.upper} have the following form:
\begin{displaymath}
 \begin{array}{lcl}
   \bfblue{bounds.phase($p$).state.lower} & = &
  \left[\begin{array}{ccc} y_{1}^{\textrm{lower}} & \cdots & y_{n_y^{(p)}}^{\textrm{lower}}\end{array}\right] \\
   \bfblue{bounds.phase($p$).state.upper} & = &
  \left[\begin{array}{ccc} y_{1}^{\textrm{upper}} & \cdots & y_{n_y^{(p)}}^{\textrm{upper}}\end{array}\right]
\end{array}
\end{displaymath}
\item \bfblue{bounds.phase($p$).finalstate.lower} and \bfblue{bounds.phase($p$).finalstate.upper}:
row vectors of length $n_y^{(p)}$ that contain the lower and upper
bounds on the final state in phase $p\in[1,\ldots,P]$.  The row
vectors \bfblue{bounds.phase($p$).finalstate.lower} and
\bfblue{bounds.phase($p$).finalstate.upper} have the following form: 
\begin{displaymath}
 \begin{array}{lcl}
   \bfblue{bounds.phase($p$).finalstate.lower} & = &
  \left[\begin{array}{ccc} y_{f,1}^{\textrm{lower}} & \cdots & y_{f,n_y^{(p)}}^{\textrm{lower}}\end{array}\right] \\
   \bfblue{bounds.phase($p$).finalstate.upper} & = &
  \left[\begin{array}{ccc} y_{f,1}^{\textrm{upper}} & \cdots & y_{f,n_y^{(p)}}^{\textrm{upper}}\end{array}\right]
\end{array}
\end{displaymath}
\item \bfblue{bounds.phase($p$).control.lower} and \bfblue{bounds.phase($p$).control.upper}:
row vectors of length $n_y^{(p)}$ that contain the lower and upper
bounds on the control during phase $p\in[1,\ldots,P]$.  The row
vectors \bfblue{bounds.phase($p$).control.lower} and
\bfblue{bounds.phase($p$).control.upper} have the following form:
\begin{displaymath}
 \begin{array}{lcl}
   \bfblue{bounds.phase($p$).control.lower} & = &
  \left[\begin{array}{ccc} u_{1}^{\textrm{lower}} & \cdots & u_{n_u^{(p)}}^{\textrm{lower}}\end{array}\right] \\
   \bfblue{bounds.phase($p$).control.upper} & = &
  \left[\begin{array}{ccc} u_{1}^{\textrm{upper}} & \cdots & u_{n_u^{(p)}}^{\textrm{upper}}\end{array}\right]
\end{array}
\end{displaymath}
\item \bfblue{bounds.phase($p$).path.lower} and \bfblue{bounds.phase($p$).path.upper}:
row vectors of length $n_c^{(p)}$ that contain the lower and upper
bounds on the path constraints during phase $p\in[1,\ldots,P]$.  The row
vectors \bfblue{bounds.phase($p$).path.lower} and
\bfblue{bounds.phase($p$).path.upper} have the following form:
\begin{displaymath}
 \begin{array}{lcl}
   \bfblue{bounds.phase($p$).path.lower} & = &
  \left[\begin{array}{ccc} c_{1}^{\textrm{lower}} & \cdots & c_{n_u^{(p)}}^{\textrm{lower}}\end{array}\right] \\
   \bfblue{bounds.phase($p$).path.upper} & = &
  \left[\begin{array}{ccc} c_{1}^{\textrm{upper}} & \cdots & c_{n_u^{(p)}}^{\textrm{upper}}\end{array}\right]
\end{array}
\end{displaymath}
\item \bfblue{bounds.phase($p$).integral.lower} and \bfblue{bounds.phase($p$).integral.upper}:
row vectors of length $n_d^{(p)}$ that contain the lower and upper
bounds on the integrals in phase $p\in[1,\ldots,P]$.  The row
vectors \bfblue{bounds.phase($p$).integral.lower} and
\bfblue{bounds.phase($p$).integral.upper} have the following form:
\begin{displaymath}
 \begin{array}{lcl}
   \bfblue{bounds.phase($p$).path.lower} & = &
  \left[\begin{array}{ccc} d_{1}^{\textrm{lower}} & \cdots & d_{n_d^{(p)}}^{\textrm{lower}}\end{array}\right] \\
   \bfblue{bounds.phase($p$).path.upper} & = &
  \left[\begin{array}{ccc} d_{1}^{\textrm{upper}} & \cdots & d_{n_d^{(p)}}^{\textrm{upper}}\end{array}\right]
\end{array}
\end{displaymath}
\item \bfblue{bounds.parameters.lower} and \bfblue{bounds.parameters.upper}:
row vectors of length $n_s$ that contain the lower and upper
bounds on the static parameters in the problem.  The row
vectors \bfblue{bounds.parameters.lower} and 
\bfblue{bounds.parameters.upper} have the following form:
\begin{displaymath}
 \begin{array}{lcl}
   \bfblue{bounds.parameters.lower} & = &
  \left[\begin{array}{ccc} s_{1}^{\textrm{lower}} & \cdots & s_{n_s}^{\textrm{lower}}\end{array}\right] \\
   \bfblue{bounds.parameters.upper} & = &
  \left[\begin{array}{ccc} s_{1}^{\textrm{upper}} & \cdots & s_{n_s}^{\textrm{upper}}\end{array}\right] \\
\end{array}
\end{displaymath}
\item \bfblue{bounds.eventgroup($g$).lower} and \bfblue{bounds.eventgroup($g$).upper}:
row vectors of length $n_b^{(g)}$ that contain the lower and upper
bounds on the group $g=1,\ldots,G$ of event constraints.  The row
vectors \\ \bfblue{bounds.eventgroup($g$).lower} and
\bfblue{bounds.eventgroup($g$).upper} have the following form:
\begin{displaymath}
 \begin{array}{lcl}
   \bfblue{bounds.eventgroup($g$).lower} & = & \left[\begin{array}{ccc} b_{1}^{\textrm{lower}} & \cdots & b_{n_b^{(g)}}^{\textrm{lower}}\end{array}\right] \\
   \bfblue{bounds.eventgroup($g$).upper} & = & \left[\begin{array}{ccc} b_{1}^{\textrm{upper}} & \cdots & b_{n_b^{(g)}}^{\textrm{upper}}\end{array}\right]
\end{array}
\end{displaymath}
\end{itemize}
{\noindent}{\bf Note:} any fields that do not apply to a problem (for
example, a problem with no path constraints) should be omitted
completely.  

\subsection{Syntax of Endpoint Function \slred{setup}.\bfblue{functions}.\bfblue{endpoint}}\label{sect:endpointSyntax}

The syntax used to evaluate the user-defined endpoint function defined
by the function handle \slred{setup}.\bfblue{functions}.\bfblue{endpoint} is given as follows:
\begin{center}
  \noindent{\bf function output=endpointfun(input)}
\end{center}
The input \slred{input} is a structure that contains the fields
\bfblue{phase}, \bfblue{auxdata}, and \bfblue{parameter} if the problem has parameters.  The field
\slred{input}.\bfblue{phase} is an array of structures of length $P$
(where $P$ is the number of phases) such that the $p^{th}$ element of 
\slred{input}.\bfblue{phase} contains the following fields:
\begin{itemize}
  \item \slred{input}.\bfblue{phase($p$).initialtime}:  a scalar that contains the initial time in phase $p=1,\ldots,P$;
 \item \slred{input}.\bfblue{phase($p$).finaltime}:  a scalar that contains the final time in phase $p=1,\ldots,P$;
 \item \slred{input}.\bfblue{phase($p$).initialstate}:  a row vector of length $n_y^{(p)}$ that contains the initial state in phase $p=1,\ldots,P$;
 \item \slred{input}.\bfblue{phase($p$).finalstate}:  a row vector of length $n_y^{(p)}$ that contains the final state in phase $p=1,\ldots,P$;
 \item \slred{input}.\bfblue{phase($p$).integral}:  a row vector of length $n_d^{(p)}$ that contains the integrals in phase $p=1,\ldots,P$;
 \item \slred{input}.\bfblue{parameter}:  a row vector of length $n_s$ that contains the static parameters in phase $p=1,\ldots,P$;
\end{itemize}
The field \slred{input}.\bfblue{auxdata} contains the same
information as the field \slred{input}.\bfblue{auxdata} that was
specified in the structure \slred{input} that was used to specify the
information for the entire problem.   The output \slred{output} is a
structure that contains the fields \bfblue{objective} and
\bfblue{eventgroup}.  The fields \slred{output}.\bfblue{objective} and
\slred{output}.\bfblue{eventgroup} are given as follows:
\begin{itemize}
  \item \slred{output}.\bfblue{objective}: a scalar that contains the
   result of computing the objective function on the current call to 
   \slred{input}.\bfblue{functions}.\bfblue{endpoint};
 \item \slred{output}.\bfblue{eventgroup}:  an array of structures of
   length $G$ (where $G$ is the number of event groups) such that the
$g^{th}$ element in \slred{output}.\bfblue{eventgroup} is a row vector
of length $n_b^{(g)}$ that contains the result of evaluating $g^{th}$
group of event constraints at the values given in the call to the
function \slred{input}.\bfblue{functions}.\bfblue{endpoint};
\end{itemize}

\subsection{Syntax for Continuous Function \slred{setup}.\bfblue{functions}.\bfblue{continuous} \label{sect:continuousSyntax}}

The syntax used to evaluate the continuous functions defined by the
function handle \slred{setup}.\bfblue{functions}.\bfblue{continuous}
is given as follows:
\begin{center}
  \noindent{\bf function output=continuousfun(input)}
\end{center}
The input \slred{input} is a structure that contains the fields
\bfblue{phase} and \bfblue{auxdata}.  The field
\slred{input}.\bfblue{phase} is an array of structures of length $P$ 
(where $P$ is the number of phases) such that the $p^{th}$ element of
\slred{input}.\bfblue{phase} contains the following fields:
\begin{itemize}
  \item \slred{input}.\bfblue{phase($p$)}.time:  a column vector of
   length $N^{(p)}$, where $N^{(p)}$ is the number of collocation points in phase $p=1,\ldots,P$.  
  \item \slred{input}.\bfblue{phase($p$)}.\bfblue{time}:  a matrix of size 
  $N^{(p)}\times n_y^{(p)}$, where $N^{(p)}$ and $n_y^{(p)}$ are,
  respectively, the number of collocation points and the 
  dimension of the state in phase $p=1,\ldots,P$;
  \item \slred{input}.\bfblue{phase($p$)}.\bfblue{control}:  a matrix
   of size $N^{(p)}\times n_u^{(p)}$, where $N^{(p)}$ and $n_u^{(p)}$ are,
 respectively, the number of collocation points and the 
 dimension of the control in phase $p=1,\ldots,P$;
  \item \slred{input}.\bfblue{phase($p$)}.\bfblue{parameter}:  a matrix
   of size $N^{(p)}\times n_s$, where $N^{(p)}$ is the number of
   collocation points in phase $p=1,\ldots,P$ and and $n_s$ is the
   dimension of the static parameter.  [{\bf Note:}  see below for
   the reason why the static parameter has a size $N^{(p)}\times n_s$];
\end{itemize}
Finally, \slred{output} is an array of structures of length $P$ (where
$P$ is the number of phases) such that the $p^{th}$ element of
\slred{output} contains the following fields:
\begin{itemize}
\item \slred{output}.\bfblue{dynamics}:  a matrix of size $N^{(p)}\times
n_y^{(p)}$, where $N^{(p)}$ and $n_y^{(p)}$ are, respectively, the
number of collocation points and the dimension of the state in phase
$p=1,\ldots,P$;   
\item \slred{output}.\bfblue{path}:  a matrix of size $N^{(p)}\times
n_c^{(p)}$, where $N^{(p)}$ and $n_c^{(p)}$ are, respectively, the
number of collocation points and the number of path constraints 
in phase $p=1,\ldots,P$;   
\item \slred{output}.\bfblue{integrand}:  a matrix of size $N^{(p)}\times
n_d^{(p)}$, where $N^{(p)}$ and $n_d^{(p)}$ are, respectively, the
number of collocation points and the number of integrals 
in phase $p=1,\ldots,P$;   
\end{itemize}

{\noindent}{\bf IMPORTANT NOTE:}  While it may seem a bit odd, the
field \slred{input}.\bfblue{phase($p$)}.\bfblue{parameter} is actually
specified as if it were phase-dependent while it actually does not
depend upon the phase because the static parameters themselves are
independent of the phase.  Furthermore, while the static parameters
are defined as a single row vector, the arrays
\slred{input}.\bfblue{phase($p$)}.\bfblue{parameter} are actually
matrices of size $N^{(p)}\times n_s$, where $N^{(p)}$ is the number of
collocation points in each phase.  The reason for making the static
parameters phase dependent and providing them as an array with
$N^{(p)}$ rows is to improve the efficiency with which the NLP
derivatives are computed.  

\subsection{Specifying an Initial Guess of The Solution \label{sect:guess}}

The field \bfblue{guess} of the user-defined structure \slred{setup} 
contains the initial guess for the problem.  The field \bfblue{guess}
is a then structure that contains the fields \bfblue{phase} and
\bfblue{parameter}.  Assume that $M^{(p)}$ is the number of values
used in the guess for the time, state, and control in phase
$p=1,\ldots,P$.  The field \slred{setup}.\bfblue{guess}.\bfblue{phase}
is an array of structures of length $P$ such that the $p^{th}$ element of
\slred{setup}.\bfblue{guess}.\bfblue{phase} contains the following
fields: 
\begin{itemize}
\item \slred{setup}.\bfblue{guess}.\bfblue{phase($p$)}.\bfblue{time}:
  a column vector of length $M^{(p)}$ in phase $p=1,\ldots,P$;
\item \slred{setup}.\bfblue{guess}.\bfblue{phase($p$)}.\bfblue{state}:
  a matrix of size $M^{(p)}\times n_y^{(p)}$, where $n_y^{(p)}$ is the
  dimension of the state in phase $p=1,\ldots,P$; 
\item \slred{setup}.\bfblue{guess}.\bfblue{phase($p$)}.\bfblue{control}:
 a matrix of size $M^{(p)}\times n_u^{(p)}$, where $n_u^{(p)}$  is the
 dimension of the control in phase $p=1,\ldots,P$;  
\item \slred{setup}.\bfblue{guess}.\bfblue{phase($p$)}.\bfblue{integral}:
 a row vector of length $n_d^{(p)}$, where $n_d^{(p)}$ is the number
 of integrals in phase $p=1,\ldots,P$; 
\item \slred{setup}.\bfblue{guess}.\bfblue{parameter}:  a row vector
  of length size $n_s$, where $n_s$ is the number of static parameters
  in the problem.  
\end{itemize}
It is noted that the column vector of time points specified in each 
phase $p=1,\ldots,P$ in the field
\slred{setup}.\bfblue{guess}.\bfblue{phase($p$)}.\bfblue{time} must be
monotonically increasing.  

\subsection{Scaling of Optimal Control Problem\label{sect:scaling}}

It is always preferable for the user to scale an optimal control
problem of interest based on knowledge and insight of the problem.  It
is understood, however, that manually scaling a problem may be an
iterative process and consume a great deal of time and effort.   While
it is beyond the scope of the software to provide a general procedure
for scaling, in an attempt to reduce the burden on the user an
automatic scaling procedure has been implemented in $\mathbb{GPOPS-II}$.
This automatic scaling procedure is based on the method provided in
\cite{Betts3} and scales both the NLP variables and NLP constraints
using the information about the optimal control problem.  First, using
the user-supplied bounds on the time, state, control, and static
parameters, the NLP variables are scaled to lie between -0.5 and 0.5.
As a result, it is essential that the user provide {\em sensible}
bounds on all quantities (that is, do {\em not} provide unreasonably
large bounds as this will result in a poorly scaled problem).  Next,
the NLP constraints are scaled to make the row norms of the Jacobians
of the functions in the optimal control problem near unity.  The
automatic scaling procedure is by no means foolproof, but it has been
found in practice to work well on many problems that otherwise would
require scaling by hand.  The advice given here is to try the
automatic scaling procedure, but not to use it for too long if it is
proving to be unsuccessful.  Finally, the automatic scaling procedure
in $\mathbb{GPOPS-II}$ can be employed simply by setting the field
\slred{setup}.\bfblue{scales}.\bfblue{method}='automatic-bounds'.  

\section{Output from an Execution of $\mathbb{GPOPS-II}$\label{sect:output}}

The output of an execution of $\mathbb{GPOPS-II}$ is the structure \slred{output},
where \slred{output} contains the following fields:
\begin{itemize}
  \item \bfblue{result}:  a structure that contains the following fields:
    \begin{itemize}
      \item \bfblue{solution}:  the optimal time, state, and control,
        in each phase and the optimal value of the static parameter
        vector.  The optimal time, state, and control are stored, 
        respectively, in the fields \bfblue{solution}.\bfblue{phase($p$)}.\bfblue{time},
        \bfblue{solution}.\bfblue{phase($p$)}.\bfblue{state}, and
        \bfblue{solution}.\bfblue{phase($p$)}.\bfblue{control}, while the 
        static parameter is stored in the field
        \bfblue{solution}.\bfblue{parameter}; 
      \item \bfblue{objective}:  the optimal value of the objective function of the optimal control problem;
      \end{itemize}
      \item \bfblue{resultsetup}:  the setup structure that produced the result found in \bfblue{result} with $\mathbb{GPOPS-II}$;
      \item \bfblue{meshhistory}:  the history of the each mesh on
        which the NLP was solved (only if mesh refinement is used);
      \item \bfblue{meshiterations}:  the number of mesh refinement
        iterations that were taken by $\mathbb{GPOPS-II}$ (only if mesh refinement is used);
\end{itemize}

\section{Useful Information for Debugging a $\mathbb{GPOPS-II}$ Problem}

One aspect of $\mathbb{GPOPS-II}$ that may appear confusing when debugging
code pertains to the dimensions of the arrays and the corresponding
time values.  It is important to remember that $\mathbb{GPOPS-II}$ uses
collocation at {\em Legendre-Gauss-Radau} points.  Because the
Legendre-Gauss-Radau points include the initial point but do not
include the final point, the dynamics, path constraints, and integrand
cost are computed only at the Legendre-Gauss-Radau points.  While this
may appear to be a bit strange, the fundamental point here is that
Legendre-Gauss-Radau quadrature (which is used in $\mathbb{GPOPS-II}$) only
evaluates the functions at the Legendre-Gauss-Radau points.  Do not
try to ``fool'' $\mathbb{GPOPS-II}$ by adding the endpoints to the computation of
the dynamics, path constraints, or integrand cost.  If you do this,
you will get an error because the dimensions are incorrect.  For a
more complete mathematical description of the collocation method used
in $\mathbb{GPOPS-II}$, see the references on the Radau pseudospectral method as
given in the bibliography at the end of this document. 

\section{$\mathbb{GPOPS-II}$ Examples}

In this Chapter we provide seven examples of using $\mathbb{GPOPS-II}$.
Each of the examples are problems that have been studied extensively
in the open literature and the solutions to these problems are well
known.  The first example is the hyper-sensitive optimal control
problem from Ref.~\citen{Rao4}.  The second example is a
multiple-stage launch vehicle ascent problem taken from 
Refs.~\citen{Benson1}, \citen{Rao:2010:TOMS}, and \citen{Betts3}.  
The third example is a tumor anti-angiogenesis optimal control problem 
from Refs.~\citen{Ledzewicz2008a} and \citen{Betts3}.  The fourth
example is the reusable launch vehicle entry problem taken from
Ref.~\citen{Betts3}.  The fifth example is the minimum
time-to-climb of a supersonic aircraft taken from
Refs.~\citen{Bryson1} and \citen{Betts3}.  The sixth example is the
optimal control of a hang glider and is taken from
Ref.~\citen{Zhao2}.  Finally, the seventh example is the optimal
control of a two-strain tuberculosis model and is taken from
Ref.~\citen{Jung02}.  For each example the optimal control problem is
described quantitatively, the $\mathbb{GPOPS-II}$ code is
provided, and the solution obtained using $\mathbb{GPOPS-II}$
is provided.   For reference, all examples were solved on a 2.5 GHz
Core i7 MacBook Pro with 16 GB of RAM running Mac OS-X 10.7.5 (Lion).
Finally, in the cases where IPOPT was used as the NLP solver, the
IPOPT MATLAB mex files available on the IPOPT website were used.
These IPOPT mex files were compiled with the linear solver MUMPS.  

\subsection{Hyper-Sensitive Problem}

Consider the following {\em hyper-sensitive} \cite{Rao3,Rao4,Rao5,Rao6}
optimal control problem adapted from Ref.~\cite{Rao4}.  Minimize the
cost functional 
\begin{equation}\label{hyperSensitiveCost}
J = \frac{1}{2}\int_{0}^{t_f} (x^2+u^2) dt
\end{equation}
subject to the dynamic constraint
\begin{equation}\label{hyperSensitiveDynamics}
  \dot{x} = -x^3+u
\end{equation}
and the boundary conditions
\begin{equation}\label{hyperSensitiveBCs}
  \begin{array}{lclclcl}
    x(0) & = & 1.5 & , & x(t_f) & = & 1
  \end{array}
\end{equation}
where $t_f$ is fixed.  It is known that for sufficiently large values
of $t_f$ that the solution to this example exhibits a so called
``take-off'', ``cruise'', and ``landing'' structure where the
interesting behavior occurs near the initial and final time (see
Ref.~\cite{Rao4} for details).  In particular, the ``cruise'' segment
of this trajectory is constant (that is, the segment where the state
and control are, interestingly, both zero) becomes an increasingly
large percentage of the total trajectory time as $t_f$ increases.
Given the structure of the solution, one would expect that the
majority of collocation points would be placed in the ``take-off'' and
``landing'' segments while few collocation points would be placed in
the ``cruise'' segment.

The hyper-sensitive optimal control problem of
Eqs.~(\ref{hyperSensitiveCost})--(\ref{hyperSensitiveBCs}) was solved
using $\mathbb{GPOPS-II}$ witht the NLP solver IPOPT and a mesh
refinement tolerance $\epsilon=10^{-7}$.  In order to solve this
problem using $\mathbb{GPOPS-II}$, the continuous function,
hyperSensitiveContinuous.m, was written to compute both the right-hand
side of the differential equations and the integrand of the Lagrange
cost.  The result of integrating the integral specified in
hyperSensitiveContinuous is then the field 'integral' of the structure
\slred{input} to the endpoint function hyperSensitiveEndpoint.m.  The
complete MATLAB code that was written to solve the hyper-sensitive
optimal control problem of
Eqs.~(\ref{hyperSensitiveCost})--(\ref{hyperSensitiveBCs}) is given
below.  
\begin{shadedframe}
\verbatiminput{../examples/hyperSensitive/hyperSensitiveMain.m}
\verbatiminput{../examples/hyperSensitive/hyperSensitiveContinuous.m}
\verbatiminput{../examples/hyperSensitive/hyperSensitiveEndpoint.m}
\end{shadedframe}
The state, $x(t)$, control, $u(t)$, and mesh refinement history that
arise from the execution of $\mathbb{GPOPS-II}$ with the above code and the NLP
solver IPOPT is summarized in
Figs.~\ref{fig:hyperSensitiveState}--\ref{fig:hyperSensitiveMeshRefinement},
while a table showing the estimate of the relative error as a function
of the mesh refinement iteration is shown in Table~\ref{tab:hyperSensitiveMeshRefinement}.
\begin{figure}[h]
\vspace*{-0.4in}
\psfragscanon
\psfrag{time}[][]{\footnotesize$t$}\psfrag{state}[][]{\footnotesize$x(t)$}\psfrag{control}[][]{\footnotesize$u(t)$}
\psfrag{Mesh Iteration}[][]{\footnotesize Mesh Refinement Iteration}
\psfrag{Mesh Point Location (Fraction of Interval)}[][]{\footnotesize Mesh Point Locations} 
\psfrag{0}[][]{\footnotesize 0}  \psfrag{1}[][]{\footnotesize 1}  \psfrag{2}[][]{\footnotesize 2}  \psfrag{3}[][]{\footnotesize 3}  \psfrag{4}[][]{\footnotesize 4}  \psfrag{5}[][]{\footnotesize 5}
\psfrag{6}[][]{\footnotesize 6}  \psfrag{7}[][]{\footnotesize 7}  \psfrag{8}[][]{\footnotesize 8}  \psfrag{9}[][]{\footnotesize 9}  \psfrag{10}[][]{\footnotesize 10}  \psfrag{11}[][]{\footnotesize 11}
\psfrag{12}[][]{\footnotesize 12}  \psfrag{13}[][]{\footnotesize 13}
\psfrag{-0.5}[][]{\footnotesize -0.5} \psfrag{-0.2}[][]{\footnotesize -0.2}
\psfrag{0.2}[][]{\footnotesize 0.2}  \psfrag{0.4}[][]{\footnotesize 0.4}  \psfrag{0.5}[][]{\footnotesize 0.5}\psfrag{0.6}[][]{\footnotesize 0.6}  \psfrag{0.8}[][]{\footnotesize 0.8}
\psfrag{1.2}[][]{\footnotesize 1.2}  \psfrag{1.4}[][]{\footnotesize 1.4}  \psfrag{1.5}[][]{\footnotesize 1.5}  \psfrag{1.6}[][]{\footnotesize 1.6}  \psfrag{0.8}[][]{\footnotesize 1.8}
\psfrag{2.5}[][]{\footnotesize 2.5}  \psfrag{1000}[][]{\footnotesize 1000}\psfrag{2000}[][]{\footnotesize 2000}\psfrag{3000}[][]{\footnotesize 3000}\psfrag{4000}[][]{\footnotesize 4000}
\psfrag{5000}[][]{\footnotesize 5000}
\centering
\hspace*{-0.5in}\subfloat[$x(t)$ vs.~$t$.\label{fig:hyperSensitiveState}]{\epsfig{figure=hyperSensitiveState.eps,height=2.7in}}~~~\subfloat[$u(t)$ vs.~$t$. \label{fig:hyperSensitiveControl}]{\epsfig{figure=hyperSensitiveControl.eps,height=2.7in}}

    \subfloat[Mesh Refinement History.\label{fig:hyperSensitiveMeshRefinement}]{\epsfig{figure=hyperSensitiveMeshRefinement.eps,height=2.7in}}

    \caption{Solution to Hyper-Sensitive Problem Obtained Using  $\mathbb{GPOPS-II}$ with the NLP Solver IPOPT and a Mesh  Refinement Tolerance of $10^{-7}$. \label{fig:hyperSensitiveSolution}}
\end{figure}

\begin{table}[h]
\centering
 \caption{Relative Error Estimate vs.~Mesh Refinement Iteration for
  Hyper-Sensitive Problem. \label{tab:hyperSensitiveMeshRefinement}}
\small
\begin{tabular}{|c|c|} \hline
{\bf Mesh Refinement Iteration} & {\bf Relative Error Estimate} \\\hline\hline
 1 & $6.0129\times 10^{-1}$ \\\hline
 2 & $1.1855\times 10^{0}$ \\\hline
 3 & $2.9973\times 10^{-1}$ \\\hline
 4 & $7.6195\times 10^{-2}$ \\\hline
 5 & $1.7983\times 10^{-2}$ \\\hline
 6 & $4.7709\times 10^{-3}$ \\\hline
 7 & $6.0427\times 10^{-4}$ \\\hline
 8 & $2.6922\times 10^{-5}$ \\\hline
 9 & $7.6867\times 10^{-5}$ \\\hline
 10 & $3.2845\times 10^{-7}$ \\\hline
 11 & $1.1056\times 10^{-7}$ \\\hline
 12 & $4.9712\times 10^{-6}$ \\\hline
 13 & $4.0017\times 10^{-8}$ \\\hline
\end{tabular}
\end{table}

\clearpage

\subsection{Multiple-Stage Launch Vehicle Ascent Problem}

The problem considered in this section is the ascent of a
multiple-stage launch vehicle.  The objective is to maneuver the
launch vehicle from the ground to the target orbit while maximizing
the remaining fuel in the upper stage.   It is noted that this example
is is found verbatim in Refs.~\citen{Benson1}, \citen{Rao:2010:TOMS},
and \citen{Betts3}.  

\subsubsection{Vehicle Properties}

The goal of this launch vehicle ascent problem is to steer the
vehicle from launch to a geostationary transfer orbit (GTO).  The
motion of the ehicle is divided into {\em four} distinct phases.
Phase 1 starts with the vehicle on the ground and terminates
when the fuel of the first set of solid rocket boosters is depleted.  Upon
termination of Phase 1 the first set of solid rocket boosters are
dropped. Phase 2 starts where Phase 1 terminates and terminates when
the fuel of the second set of solid rockets boosters is depleted.
Phase 3 starts when Phase 2 terminates and termintes when the fuel of
the first main engine fuel is depleted.  Finally, Phase 4 starts where
Phase 3 terminates and terminates when the vehicle reaches the final
GTO.  The vehicle data for this problem is taken verbatim from
Ref.~\citen{Rao:2010:TOMS} or \citen{Betts3} and is shown in Table
\ref{table: launch vehicle properties}. 

\begin{table}[htdp]
\centering
\caption{Vehicle Properties for Multiple-Stage Launch Vehicle Ascent Problem. \label{table: launch vehicle properties}}
\begin{tabular}{|c|c|c|c|}
\hline
 & Solid Motors & Stage 1 & Stage 2 \\
 \hline \hline
 Total Mass (kg) & 19290 & 104380 & 19300 \\
 \hline
 Propellant Mass (kg) & 17010 & 95550 & 16820 \\
 \hline
 Engine Thrust (N) & 628500 & 1083100 & 110094 \\
 \hline
 Isp (sec) & 284 & 301.7 & 462.4 \\
 \hline
 Number of Engines & 9 & 1 & 1 \\
 \hline
 Burn Time (sec) & 75.2 & 261 & 700 \\
 \hline
\end{tabular}
\end{table}

\subsubsection{Dynamic Model}

The equations of motion for a non-lifting point mass in flight over a
spherical rotating planet are expressed in Cartesian Earth centered inertial
(ECI) coordinates as
\begin{equation}\label{dyncs}
\begin{array}{rcl}
  \dot{\textbf{r}} &=& \mathbf{v} \vspace{3pt}\\
  \dot{\textbf{v}} &=& -\displaystyle\frac{\mu}{\|\textbf{r}\|^3}\mathbf{r} +
  \displaystyle\frac{T}{m}\mathbf{u} + \displaystyle\frac{\mathbf{D}}{m}  \vspace{3pt}\\
  \dot{m} & = & -\displaystyle\frac{T}{g_0I_{sp}}
\vspace{3pt}\\
\end{array}
\end{equation}
where $\mathbf{r}(t)=\left[\begin{array}{ccc} x(t) & y(t) & z(t)\end{array}\right]^T$
is the position, $\mathbf{v} = \left[\begin{array}{ccc} v_x(t) & v_y(t) & v_z(t)\end{array}\right]^T$
is the Cartesian ECI velocity, $\mu$ is the gravitational parameter, $T$ is
the vacuum thrust, $m$ is the mass, $g_0$ is the acceleration due to gravity at sea level,
$I_{sp}$ is the specific impulse of the engine,
$\mathbf{u} = \left[\begin{array}{ccc} u_x & u_y & u_z \end{array}\right]^T$ is the thrust
direction, and $\mathbf{D}=\left[\begin{array}{ccc} D_x & D_y & D_z \end{array}\right]^T$
is the drag force.  The drag force is defined as
\begin{equation}
  \mathbf{D} = -\frac{1}{2}C_D A_{ref}\rho \|\mathbf{v}_{\textrm{rel}}\|\mathbf{v}_{\textrm{rel}}
\end{equation}
where $C_D$ is the drag coefficient, $A_{ref}$ is the reference area, $\rho$
is the atmospheric density, and $\mathbf{v}_{\textrm{rel}}$ is the Earth relative
velocity, where $\mathbf{v}_{\textrm{rel}}$ is given as
\begin{equation}
\mathbf{v}_{\textrm{rel}} = \mathbf{v}-\boldsymbol{\omega} \times \mathbf{r}
\end{equation}
where $\boldsymbol\omega$ is the angular velocity of the Earth relative to
inertial space.  The atmospheric density is modeled as the exponential
function
\begin{equation}
\rho = \rho_0\mbox{exp}[-h/h_0]
\end{equation}
where $\rho_0$ is the atmospheric density at sea level, $h=\|\bfr\|-R_e$ is
the altitude, $R_e$ is the equatorial radius of the Earth, and $h_0$ is the
density scale height.  The numerical values for these constants can be found
in Table \ref{dynamics properties}.

\begin{table}[htdp]
\caption{Constants used in the launch vehicle example.}
\begin{center}
\begin{tabular}{|c|c|}
\hline
Constant & Value \\
\hline \hline
Payload Mass (kg) & 4164 \\
\hline
$A_{ref}$ (m${}^2$) & $4\pi$ \\
\hline
$C_d$ & 0.5 \\
\hline
$\rho_0$ (kg/m${}^3$)& 1.225 \\
\hline
$h_0$ (km) & 7.2\\
\hline
$t_1$ (s) & 75.2 \\
\hline
 $t_2$ (s) & 150.4 \\
\hline
 $t_3$ (s) & 261 \\
\hline
 $R_e$ (km) & 6378.14 \\
\hline
 $V_E$ (km/s) & 7.905\\
\hline
\end{tabular}
\end{center}
\label{dynamics properties}
\end{table}

\subsubsection{Constraints}

The launch vehicle starts on the ground at rest (relative to the Earth) at time $t_0$, so that the ECI initial conditions are
\begin{equation}\label{ICs}
\begin{array}{rcl}
\mathbf{r}(t_0) &=& \mathbf{r}_0 = \left[ \begin{array}{ccc} 5605.2 & 0 & 3043.4 \end{array} \right] ^T\quad \mbox{km} \vspace{3pt}\\
\mathbf{v}(t_0) &=& \mathbf{v}_0 = \left[ \begin{array}{ccc} 0 & 0.4076 & 0 \end{array} \right]^T \quad \mbox{km/s} \vspace{3pt}\\
m(t_0) &=& m_0 = 301454 \quad \mbox{kg}
\end{array}
\end{equation}
The terminal constraints define the target geosynchronous transfer
orbit (GTO), which is defined in orbital elements as
\begin{equation}\label{FCs}
\begin{array}{rcl}
 a_f &=   &  24361.14 \; \mbox{km}, \\
 e_f &=   &  0.7308, \\
 i_f &=   &  28.5\deg,\\
 \Omega_f &= & 269.8\deg, \\
 \omega_f &= & 130.5\deg
\end{array}
\end{equation}
The orbital elements, $a,e,i,\Omega$, and $\omega$ represent the semi-major
axis, eccentricity, inclination, right ascension of the ascending node
(RAAN), and argument of perigee, respectively.  Note that the true anomaly,
$\nu$, is left undefined since the exact location within the orbit is not
constrained.  These orbital elements can be transformed into ECI coordinates
via the transformation, $T_{o2c}$, where $T_{o2c}$ is given in \cite{Bate1}.

In addition to the boundary constraints, there exists both a state path
constraint and a control path constraint in this problem.  A state path
constraint is imposed to keep the vehicle's altitude above the surface of the
Earth, so that
\begin{equation}\label{xpath}
|\mathbf{r}|\geq R_r
\end{equation}
where $R_e$ is the radius of the Earth, as seen in Table \ref{dynamics properties}.  Next, a path constraint is imposed on the control to guarantee
that the control vector is unit length, so that
\begin{equation}\label{upath}
  \|\mathbf{u}\|_2^2 = u_1^2 + u_2^2 + u_3^2 = 1
\end{equation}

Lastly, each of the four phases in this trajectory is linked to the
adjoining phases by a set of linkage conditions.  These constraints
force the position and velocity to be continuous and also account for
the mass ejections, as 
\begin{equation}
\begin{array}{rcl}
\mathbf{r}^{(p)}(t_f)-\mathbf{r}^{(p+1)}(t_0) &=& \mathbf{0}, \\
\mathbf{v}^{(p)}(t_f)-\mathbf{v}^{(p+1)}(t_0) &=& \mathbf{0}, \qquad (p=1,\ldots,3)\\
m^{(p)}(t_f)-m_{dry}^{(p)}-m^{(p+1)}(t_0) &=& 0 \\
\end{array}
\end{equation}
where the superscript $(p)$ represents the phase number.

The optimal control problem is then to find the control, $\mathbf{u}$,
that minimizes the cost function
\begin{equation}
  J=-m^{(4)}(t_f)
\end{equation}
subject to the conditions of Eqs.~(\ref{dyncs}), (\ref{ICs}), (\ref{FCs}),
(\ref{xpath}), and (\ref{upath}).

The MATLAB code that solves the multiple-stage launch vehicle ascent
problem using $\mathbb{GPOPS-II}$ is shown below.  In particular, this
problem requires the specification of a function that computes the
cost functional, the differential-algebraic equations (which, it is
noted, include both the differential equations {\em and} the path
constraints), and the event constraints in each phase of the problem
along with the phase-connect (\ie linkage) constraints.  The problem
was posed in SI units and the built-in autoscaling procedure was
used.
\begin{shadedframe}
\verbatiminput{../examples/launch/launchMain.m}
\verbatiminput{../examples/launch/launchContinuous.m}
\verbatiminput{../examples/launch/launchEndpoint.m}
\verbatiminput{../examples/launch/launchEvent.m}
\verbatiminput{../examples/launch/launchrv2oe.m}
\verbatiminput{../examples/launch/launchoe2rv.m}
\end{shadedframe}  
The output of the above code from $\mathbb{GPOPS-II}$ is summarized in
the following three plots that contain the altitude, speed, and
controls.

\begin{figure}[h]
 \psfragscanon
 \psfrag{time (s)}[][]{\footnotesize$t$ (s)}\psfrag{altitude (km)}[][]{\footnotesize$h(t)$ (km)}\psfrag{control}[][]{\footnotesize$\mathbf{u}(t)$}
 \psfrag{speed (m/s)}[][]{\footnotesize$v(t)$ (m/s)}
 \psfrag{Mesh Iteration}[][]{\footnotesize Mesh Refinement Iteration}
 \psfrag{Mesh Point Location (Fraction of Interval)}[][]{\footnotesize Mesh Point Locations} 
 \psfrag{-0.8}[][]{\footnotesize -0.8}\psfrag{-0.6}[][]{\footnotesize -0.6}\psfrag{-0.4}[][]{\footnotesize -0.4}\psfrag{-0.2}[][]{\footnotesize -0.2}
 \psfrag{0.2}[][]{\footnotesize 0.2}\psfrag{0.4}[][]{\footnotesize 0.4}\psfrag{0.6}[][]{\footnotesize 0.6}\psfrag{0.8}[][]{\footnotesize 0.8}
 \psfrag{50}[][]{\footnotesize 50}\psfrag{100}[][]{\footnotesize 100}\psfrag{150}[][]{\footnotesize 150}
 \psfrag{0}[][]{\footnotesize 0}\psfrag{1}[][]{\footnotesize 1}\psfrag{200}[][]{\footnotesize 200}\psfrag{250}[][]{\footnotesize 250}\psfrag{400}[][]{\footnotesize 400}  
 \psfrag{600}[][]{\footnotesize 600}   \psfrag{800}[][]{\footnotesize 800}   \psfrag{1000}[][]{\footnotesize 1000}  
 \psfrag{2000}[][]{\footnotesize 2000}\psfrag{4000}[][]{\footnotesize 4000}\psfrag{6000}[][]{\footnotesize 6000}
 \psfrag{8000}[][]{\footnotesize 8000}\psfrag{10000}[][]{\footnotesize 10000}\psfrag{12000}[][]{\footnotesize 12000}
 \centering
 \hspace*{-0.5in}\subfloat[$h(t)$ vs.~$t$.\label{fig:launchAltitude}]{\epsfig{figure=launchAltitude.eps,height=2.7in}}~~~\subfloat[$v(t)$ vs.~$t$. \label{fig:launchSpeed}]{\epsfig{figure=launchSpeed.eps,height=2.7in}}

\subfloat[$\mathbf{u}(t)$ vs.~$t$.\label{fig:launchControl}]{\epsfig{figure=launchControl.eps,height=2.7in}}

    \caption{Solution to Launch Vehicle Ascent Problem Using $\mathbb{GPOPS-II}$ with the NLP Solver SNOPT and a Mesh Refinement Tolerance of $10^{-7}$. \label{fig:launchSolution}}

\end{figure}

\clearpage

\subsection{Tumor-Antiangiogenesis Optimal Control Problem}

Consider the following cancer treatment optimal control problem taken
from Ref.~\citen{Ledzewicz2008a}.  The objective is to minimize
\begin{equation}\label{tumorCost}
  p(t_f)
\end{equation}
subject to the dynamic constraints
\begin{equation}\label{tumorDynamics}
  \begin{array}{lcl}
    \dot{p}(t) & = & -\xi p(t) \ln\left(\frac{p(t)}{q(t)}\right), \\
    \dot{q}(t) & = & q(t) \left[ b - \mu - d p^{2/3}(t) - Gu(t) \right],
 \end{array}
\end{equation}
with the initial conditions
\begin{equation} \label{tumorICs}
\begin{array}{lcl}
  p(0) & = & p_0, \\
  q(0) & = & q_0, \\
  \end{array}
\end{equation}
and the integral constraint
\begin{equation}\label{tumorIntegral}
  \int_0^{t_f} u(\tau) d\tau \leq A.
\end{equation}
This problem describes a treatment process called anti-angiogenesis
where it is desired to reverse the direction of growth of a tumor by
cutting of the blood supply to the tumor.  The code for solving this
problem is shown below. 
\begin{shadedframe}
\verbatiminput{../examples/tumorAntiAngiogenesis/tumorAntiAngiogenesisMain.m}
\verbatiminput{../examples/tumorAntiAngiogenesis/tumorAntiAngiogenesisContinuous.m}
\verbatiminput{../examples/tumorAntiAngiogenesis/tumorAntiAngiogenesisEndpoint.m}
\end{shadedframe} 
The solution obtained using $\mathbb{GPOPS-II}$ using the NLP solver
IPOPT with a mesh refinement error tolerance of $10^{-6}$ is shown in
Figs.~\ref{fig:tumorAntiAngiogenesisState}--\ref{fig:tumorAntiAngiogenesisCostate}.
Note that in this example we have also provided the costate of the
optimal control problem, where the costate is estimated using the
Radau pseudospectral costate estimation method described in
Refs.~\citen{Garg1,Garg2}, and \citen{Garg3}.
\begin{figure}[h]
 \psfragscanon
 \psfrag{time}[][]{\footnotesize$t$}\psfrag{state}[][]{\footnotesize$(p(t),q(t))$}\psfrag{control}[][]{\footnotesize$u(t)$}\psfrag{costate}[][]{\footnotesize$(\lambda_p(t),\lambda_q(t))$}
 \psfrag{Mesh Iteration}[][]{\footnotesize Mesh Refinement Iteration}
 \psfrag{Mesh Point Location (Fraction of Interval)}[][]{\footnotesize Mesh Point Locations (Scaled to Interval $[0,t/t_f]$)} 
 \psfrag{-0.8}[][]{\footnotesize -0.8}\psfrag{-0.6}[][]{\footnotesize -0.6}\psfrag{-0.4}[][]{\footnotesize -0.4}\psfrag{-0.2}[][]{\footnotesize -0.2}
 \psfrag{0.1}[][]{\footnotesize 0.1}\psfrag{0.2}[][]{\footnotesize 0.2}\psfrag{0.3}[][]{\footnotesize 0.3}\psfrag{0.4}[][]{\footnotesize 0.4}\psfrag{0.5}[][]{\footnotesize 0.5}
\psfrag{0.6}[][]{\footnotesize 0.6}\psfrag{0.7}[][]{\footnotesize 0.7}\psfrag{0.8}[][]{\footnotesize 0.8}\psfrag{0.9}[][]{\footnotesize 0.9}\psfrag{1}[][]{\footnotesize 1}
 \psfrag{2}[][]{\footnotesize 2}\psfrag{3}[][]{\footnotesize 3}\psfrag{4}[][]{\footnotesize 4}\psfrag{5}[][]{\footnotesize 5}\psfrag{6}[][]{\footnotesize 6}
\psfrag{7}[][]{\footnotesize 7}\psfrag{8}[][]{\footnotesize 8}\psfrag{9}[][]{\footnotesize 9}

\psfrag{1.2}[][]{\footnotesize 1.2}\psfrag{1.4}[][]{\footnotesize 1.4}
 \psfrag{10}[][]{\footnotesize 10}\psfrag{20}[][]{\footnotesize 20}\psfrag{30}[][]{\footnotesize 30}\psfrag{40}[][]{\footnotesize 40}\psfrag{50}[][]{\footnotesize 50}
 \psfrag{60}[][]{\footnotesize 60}\psfrag{70}[][]{\footnotesize 70}\psfrag{80}[][]{\footnotesize 80}
 \psfrag{0}[][]{\footnotesize 0}\psfrag{1}[][]{\footnotesize 1}\psfrag{200}[][]{\footnotesize 200}\psfrag{250}[][]{\footnotesize 250}\psfrag{400}[][]{\footnotesize 400}  
 \psfrag{600}[][]{\footnotesize 600}   \psfrag{800}[][]{\footnotesize 800}   \psfrag{1000}[][]{\footnotesize 1000}  
 \psfrag{2000}[][]{\footnotesize 2000}\psfrag{3000}[][]{\footnotesize 3000}\psfrag{4000}[][]{\footnotesize 4000}\psfrag{5000}[][]{\footnotesize 5000}
 \psfrag{6000}[][]{\footnotesize 6000}\psfrag{7000}[][]{\footnotesize 7000}\psfrag{8000}[][]{\footnotesize 8000}\psfrag{9000}[][]{\footnotesize 9000}
 \centering
 \hspace*{-0.5in}\subfloat[$(p(t),q(t))$ vs.~$t$.\label{fig:tumorAntiAngiogenesisState}]{\epsfig{figure=tumorAntiAngiogenesisState.eps,height=2.7in}}~~~\subfloat[$u(t)$ vs.~$t$. \label{fig:tumorAntiAngiogenesisControl}]{\epsfig{figure=tumorAntiAngiogenesisControl.eps,height=2.7in}}

 \hspace*{-0.5in}\subfloat[$(\lambda_p(t),\lambda_q(t),\lambda_y(t))$ vs.~$t$.\label{fig:tumorAntiAngiogenesisCostate}]{\epsfig{figure=tumorAntiAngiogenesisCostate.eps,height=2.7in}}~~~\subfloat[Mesh Refinement History.\label{fig:tumorAntiAngiogenesisMeshRefinement}]{\epsfig{figure=tumorAntiAngiogenesisMeshRefinement.eps,height=2.7in}}

    \caption{Solution to Tumor Anti-Angiogenesis Optimal Control
      Problem Using $\mathbb{GPOPS-II}$ with the NLP Solver SNOPT and a Mesh Refinement Tolerance of $10^{-6}$. \label{fig:tumorAntiAngiogenesisSolution}}

\end{figure}

\clearpage

\subsection{Reusable Launch Vehicle Entry}

Consider the following optimal control problem of maximizing the
crossrange during the atmospheric entry of a reusable  launch
vehicle and taken verbatim from Ref.~\cite{Betts3}.  Minimize the cost
functional
\begin{equation}\label{rlv cost}
  J = -\phi(t_f) 
\end{equation}
subject to the dynamic constraints
\begin{equation}\label{rlv eom}
  \begin{array}{lclclclcl}
    \dot{r} & = & v \sin \gamma, \vspace{3pt} \\ 
    \dot{\theta} & = & \displaystyle \frac{v \cos \gamma \sin \psi}{r \cos
      \phi},  \vspace{3pt} \\ 
    \dot{\phi} & = & \displaystyle \frac{v \cos \gamma \cos \psi}{r}, \\ 
    \dot{v} & = & \displaystyle -\frac{F_d}{m}-F_g\sin \gamma, \vspace{3pt} \\ 
    \dot{\gamma} & = & \displaystyle \frac{F_l \cos \sigma}{ m v} -\left(\frac{F_g}{v}-\frac{v}{r}\right)\cos\gamma , \vspace{3pt} \\ 
    \dot{\psi}  & = & \displaystyle \frac{F_l \sin \sigma}{m v \cos \gamma}+\frac{v \cos \gamma \sin \psi \tan \phi }{r},  
  \end{array}
\end{equation}
and the boundary conditions
\begin{equation} \label{rlv bcs}
  \begin{array}{lclclcl}
    r(0) & = & 79248+R_e \textrm{ m} & , & r(t_f) & = & 24384+R_e \textrm{ m}, \\
    \theta(0) & = & 0 \textrm{ deg} & , & \theta(t_f) & = & \textrm{Free}, \\
    \phi(0) & = & 0 \textrm{ deg} & , & \phi(t_f) & = & \textrm{Free},  \\
    v(0) & = & 7803 \textrm{ m/s} & , & v(t_f) & = & 762 \textrm{ m/s}, \\
    \gamma(0) & = & -1 \textrm{ deg} & , & \gamma(t_f) & = & -5 \textrm{ deg},
 \\    \psi(0) & = & 90 \textrm{ deg} & , & \psi(t_f) & = & \textrm{Free}. \\
  \end{array}
\end{equation}
Further details of this problem, including the aerodynamic model, can
be found in Ref.~\citen{Betts3}.  The code for solving this problem is
shown below.  
\begin{shadedframe}
\verbatiminput{../examples/rlvEntry/rlvEntryMain.m}
\verbatiminput{../examples/rlvEntry/rlvEntryContinuous.m}
\verbatiminput{../examples/rlvEntry/rlvEntryEndpoint.m}
\end{shadedframe} 
This example was solved using $\mathbb{GPOPS-II}$ using the NLP solver
IPOPT with a mesh refinement tolerance of $10^{-6}$ and the solution
is shown in
Figs.~\ref{fig:rlvEntryAltitude}--\ref{fig:rlvEntryBankAngle}.  
\begin{figure}[h]
\vspace*{-0.4in}
\psfragscanon
\psfrag{Time (s)}[][]{\footnotesize$t$ (s)}\psfrag{Altitude (km)}[][]{\footnotesize$h(t)$ (km)}\psfrag{Speed (km/s)}[][]{\footnotesize$v(t)$ (km/s)}
\psfrag{Latitude (deg)}[][]{\footnotesize$\phi(t)$ (deg)}\psfrag{Longitude (deg)}[][]{\footnotesize$\theta(t)$ (deg)}\psfrag{Flight Path Angle (deg)}[][]{\footnotesize$\gamma(t)$ (deg)}
\psfrag{Angle of Attack (deg)}[][]{\footnotesize$\alpha(t)$ (deg)}\psfrag{Bank Angle (deg)}[][]{\footnotesize$\sigma(t)$ (deg)}
\psfrag{Mesh Iteration}[][]{\footnotesize Mesh Refinement Iteration}
\psfrag{Mesh Point Location (Fraction of Interval)}[][]{\footnotesize Mesh Point Locations} 
\psfrag{16.5}[][]{\footnotesize 16.5}\psfrag{17}[][]{\footnotesize 17}\psfrag{17.5}[][]{\footnotesize 17.5}
\psfrag{-10}[][]{\footnotesize -10}\psfrag{-20}[][]{\footnotesize -20}\psfrag{-30}[][]{\footnotesize -30}\psfrag{-40}[][]{\footnotesize -40}
\psfrag{-50}[][]{\footnotesize -50}\psfrag{-60}[][]{\footnotesize -60}\psfrag{-70}[][]{\footnotesize -70}\psfrag{-80}[][]{\footnotesize -80}
\psfrag{0}[][]{\footnotesize 0}\psfrag{1}[][]{\footnotesize 1}\psfrag{2}[][]{\footnotesize 2}\psfrag{3}[][]{\footnotesize 3}\psfrag{4}[][]{\footnotesize 4}
\psfrag{5}[][]{\footnotesize 5}\psfrag{10}[][]{\footnotesize 10}\psfrag{15}[][]{\footnotesize 15}\psfrag{20}[][]{\footnotesize 20}
\psfrag{25}[][]{\footnotesize 25}\psfrag{30}[][]{\footnotesize 30}\psfrag{35}[][]{\footnotesize 35}\psfrag{40}[][]{\footnotesize 40}\psfrag{45}[][]{\footnotesize 45}
\psfrag{50}[][]{\footnotesize 50}\psfrag{55}[][]{\footnotesize 55}\psfrag{60}[][]{\footnotesize 60}\psfrag{65}[][]{\footnotesize 65}
\psfrag{70}[][]{\footnotesize 70}\psfrag{75}[][]{\footnotesize 75}\psfrag{80}[][]{\footnotesize 80}
\psfrag{500}[][]{\footnotesize 500}\psfrag{1000}[][]{\footnotesize 1000}\psfrag{1500}[][]{\footnotesize 1500}\psfrag{2000}[][]{\footnotesize 2000}
\psfrag{2500}[][]{\footnotesize 2500}\psfrag{3000}[][]{\footnotesize 3000}\psfrag{3500}[][]{\footnotesize 3500}\psfrag{4000}[][]{\footnotesize 4000}
\psfrag{4500}[][]{\footnotesize 4500}\psfrag{5000}[][]{\footnotesize 5000}\psfrag{5500}[][]{\footnotesize 5500}\psfrag{6000}[][]{\footnotesize 6000}
\psfrag{6500}[][]{\footnotesize 6500}\psfrag{7000}[][]{\footnotesize 7000}\psfrag{7500}[][]{\footnotesize 7500}\psfrag{8000}[][]{\footnotesize 8000}
\psfrag{6}[][]{\footnotesize 6}  \psfrag{7}[][]{\footnotesize 7}  \psfrag{8}[][]{\footnotesize 8}  \psfrag{9}[][]{\footnotesize 9}  \psfrag{10}[][]{\footnotesize 10}  \psfrag{11}[][]{\footnotesize 11}
\psfrag{12}[][]{\footnotesize 12}  \psfrag{13}[][]{\footnotesize 13}
\psfrag{-0.5}[][]{\footnotesize -0.5} \psfrag{-0.2}[][]{\footnotesize -0.2}
\psfrag{0.2}[][]{\footnotesize 0.2}  \psfrag{0.4}[][]{\footnotesize 0.4}  \psfrag{0.5}[][]{\footnotesize 0.5}\psfrag{0.6}[][]{\footnotesize 0.6}  \psfrag{0.8}[][]{\footnotesize 0.8}
\psfrag{1.2}[][]{\footnotesize 1.2}  \psfrag{1.4}[][]{\footnotesize 1.4}  \psfrag{1.5}[][]{\footnotesize 1.5}  \psfrag{1.6}[][]{\footnotesize 1.6}  \psfrag{0.8}[][]{\footnotesize 1.8}
\psfrag{2.5}[][]{\footnotesize 2.5}  \psfrag{1000}[][]{\footnotesize 1000}\psfrag{2000}[][]{\footnotesize 2000}\psfrag{3000}[][]{\footnotesize 3000}\psfrag{4000}[][]{\footnotesize 4000}
\psfrag{5000}[][]{\footnotesize 5000}
\centering
\hspace*{-0.5in}\subfloat[$h(t)$ vs.~$t$.\label{fig:rlvEntryAltitude}]{\epsfig{figure=rlvAltitude.eps,height=2.7in}}~~~\subfloat[$v(t)$ vs.~$t$. \label{fig:rlvEntrySpeeed}]{\epsfig{figure=rlvSpeed.eps,height=2.7in}}

\hspace*{-0.5in}\subfloat[$\phi(t)$ vs.~$\theta(t)$.\label{fig:rlvEntryLonLat}]{\epsfig{figure=rlvLonLat.eps,height=2.7in}}~~~\subfloat[$\gamma(t)$ vs.~$t$. \label{fig:rlvEntryFlightPathAngle}]{\epsfig{figure=rlvFlightPathAngle.eps,height=2.7in}}

\hspace*{-0.5in}\subfloat[$\alpha(t)$ vs.~$t$.\label{fig:rlvEntryAngleofAttack}]{\epsfig{figure=rlvAngleofAttack.eps,height=2.7in}}~~~\subfloat[$\sigma(t)$ vs.~$t$. \label{fig:rlvEntryBankAngle}]{\epsfig{figure=rlvBankAngle.eps,height=2.7in}}

    \caption{Solution to Reusable Launch Vehicle Entry Problem Using  $\mathbb{GPOPS-II}$ with the NLP Solver IPOPT and a Mesh  Refinement Tolerance of $10^{-6}$. \label{fig:rlvEntrySolution}}
\end{figure}

\clearpage

\subsection{Minimum Time-to-Climb of a Supersonic Aircraft}

The problem considered in this section is the classical minimum
time-to-climb of a supersonic aircraft.  The objective is to determine
the minimum-time trajectory and control from take-off to a specified
altitude and speed.  This problem was originally stated in the open
literature in the work of Ref.~\citen{Bryson2}, but the model used in
this study was taken from Ref.~\citen{Betts3} with the exception that
a linear extrapolation of the thrust data as found in
Ref.~\citen{Betts3} was performed in order to fill in the ``missing'' 
data points.

The minimum time-to-climb problem for a supersonic aircraft is posed
as follows.  Minimize the cost functional
\begin{equation}
  J = t_f
\end{equation}
subject to the dynamic constraints
\begin{eqnarray}
  \dot{h} & = & v\sin\alpha \\
  \dot{v} & = & \frac{T\cos\alpha-D}{m} \\
 \dot{\gamma} & = & \frac{T\sin\alpha+L}{mv}+\left(\frac{v}{r}-\frac{\mu}{vr^2}\right)\cos\gamma\\
 \dot{m} & =&  -\frac{T}{g_0 I_{sp}} 
\end{eqnarray}
and the boundary conditions
\begin{eqnarray}
  h(0) & = & 0 \textrm{ ft} \\
  v(0) & = & 129.3144 \textrm{ m/s} \\
  \gamma(0) & = & 0 \textrm{ rad} \\
  h(t_f) & = & 19994.88 \textrm{ m} \\
  v(t_f) & = & 295.092 \textrm{ ft/s} \\
  \gamma(t_f) & = & 0 \textrm{ rad}
\end{eqnarray}
where $h$ is the altitude, $v$ is the speed, $\gamma$ is the
flight path angle, $m$ is the vehicle mass, $T$ is the magnitude of
the thrust force, and $D$ is the magnitude of the drag force.  It is
noted that this example uses table data obtained from
Ref.~\citen{Bryson2}.  In this example $\mathbb{GPOPS-II}$ is implemented using the
finite-difference derivative option (that is, using the option
\slred{setup}.\bfblue{derivatives}='finite-difference') together with an interpolation of the
table data with the MATLAB functions 'interp1' and 'interp2'.   The
MATLAB code that solves the minimum time-to-climb of a supersonic 
aircraft is shown below.
\begin{shadedframe}
\verbatiminput{../examples/minimumTimeToClimb/minimumTimeToClimbMain.m}
\verbatiminput{../examples/minimumTimeToClimb/minimumTimeToClimbContinuous.m}
\verbatiminput{../examples/minimumTimeToClimb/minimumTimeToClimbEndpoint.m}
\verbatiminput{../examples/minimumTimeToClimb/minimumTimeToClimbCompute.m}
\end{shadedframe} 
The components of the state and the control obtained from running the
above $\mathbb{GPOPS-II}$ code is summarized in Figs.~\ref{fig:minimumTimeToClimbAltitude}--\ref{fig:minimumTimeToClimbAngleofAttack}.

\begin{figure}[h]
\psfragscanon
\psfrag{Time (s)}[][]{\footnotesize$t$ (s)}\psfrag{Altitude (km)}[][]{\footnotesize$h(t)$ (km)}\psfrag{Speed (km/s)}[][]{\footnotesize$v(t)$ (km/s)}
\psfrag{Flight Path Angle (deg)}[][]{\footnotesize$\gamma(t)$ (deg)}\psfrag{Angle of Attack (deg)}[][]{\footnotesize$\alpha(t)$ (deg)}
\psfrag{Mesh Iteration}[][]{\footnotesize Mesh Refinement Iteration}
\psfrag{Mesh Point Location (Fraction of Interval)}[][]{\footnotesize Mesh Point Locations (Scaled to Interval $[0,t/t_f]$)} 
\psfrag{16.5}[][]{\footnotesize 16.5}\psfrag{17}[][]{\footnotesize 17}\psfrag{17.5}[][]{\footnotesize 17.5}\psfrag{-5}[][]{\footnotesize -5}
\psfrag{-10}[][]{\footnotesize -10}\psfrag{-20}[][]{\footnotesize -20}\psfrag{-30}[][]{\footnotesize -30}\psfrag{-40}[][]{\footnotesize -40}
\psfrag{-50}[][]{\footnotesize -50}\psfrag{-60}[][]{\footnotesize -60}\psfrag{-70}[][]{\footnotesize -70}\psfrag{-80}[][]{\footnotesize -80}
\psfrag{-6}[][]{\footnotesize -6}\psfrag{-4}[][]{\footnotesize -4}\psfrag{-2}[][]{\footnotesize -2}\psfrag{2}[][]{\footnotesize 2}\psfrag{4}[][]{\footnotesize 4}
\psfrag{0}[][]{\footnotesize 0}\psfrag{5}[][]{\footnotesize 5}\psfrag{10}[][]{\footnotesize 10}\psfrag{15}[][]{\footnotesize 15}\psfrag{20}[][]{\footnotesize 20}
\psfrag{25}[][]{\footnotesize 25}\psfrag{30}[][]{\footnotesize 30}\psfrag{35}[][]{\footnotesize 35}\psfrag{40}[][]{\footnotesize 40}\psfrag{45}[][]{\footnotesize 45}
\psfrag{50}[][]{\footnotesize 50}\psfrag{55}[][]{\footnotesize 55}\psfrag{60}[][]{\footnotesize 60}\psfrag{65}[][]{\footnotesize 65}
\psfrag{70}[][]{\footnotesize 70}\psfrag{75}[][]{\footnotesize 75}\psfrag{80}[][]{\footnotesize 80}\psfrag{85}[][]{\footnotesize 85}\psfrag{90}[][]{\footnotesize 90}
\psfrag{95}[][]{\footnotesize 95}\psfrag{100}[][]{\footnotesize 100}\psfrag{150}[][]{\footnotesize 150}\psfrag{200}[][]{\footnotesize 200}
\psfrag{250}[][]{\footnotesize 250}\psfrag{300}[][]{\footnotesize 300}\psfrag{350}[][]{\footnotesize 350}\psfrag{400}[][]{\footnotesize 400}
\psfrag{450}[][]{\footnotesize 450}\psfrag{500}[][]{\footnotesize 500}
\psfrag{500}[][]{\footnotesize 500}\psfrag{1000}[][]{\footnotesize 1000}\psfrag{1500}[][]{\footnotesize 1500}\psfrag{2000}[][]{\footnotesize 2000}
\psfrag{2500}[][]{\footnotesize 2500}\psfrag{3000}[][]{\footnotesize 3000}\psfrag{3500}[][]{\footnotesize 3500}\psfrag{4000}[][]{\footnotesize 4000}
\psfrag{4500}[][]{\footnotesize 4500}\psfrag{5000}[][]{\footnotesize 5000}\psfrag{5500}[][]{\footnotesize 5500}\psfrag{6000}[][]{\footnotesize 6000}
\psfrag{6500}[][]{\footnotesize 6500}\psfrag{7000}[][]{\footnotesize 7000}\psfrag{7500}[][]{\footnotesize 7500}\psfrag{8000}[][]{\footnotesize 8000}
\psfrag{6}[][]{\footnotesize 6}  \psfrag{7}[][]{\footnotesize 7}  \psfrag{8}[][]{\footnotesize 8}  \psfrag{9}[][]{\footnotesize 9}  \psfrag{10}[][]{\footnotesize 10}  \psfrag{11}[][]{\footnotesize 11}
\psfrag{12}[][]{\footnotesize 12}  \psfrag{13}[][]{\footnotesize 13}
\psfrag{-0.5}[][]{\footnotesize -0.5} \psfrag{-0.2}[][]{\footnotesize -0.2}\psfrag{0.1}[][]{\footnotesize 0.1}
\psfrag{0.2}[][]{\footnotesize 0.2}\psfrag{0.3}[][]{\footnotesize 0.3}  \psfrag{0.4}[][]{\footnotesize 0.4}  \psfrag{0.5}[][]{\footnotesize 0.5}\psfrag{0.6}[][]{\footnotesize 0.6}  \psfrag{0.8}[][]{\footnotesize 0.8}
\psfrag{1.2}[][]{\footnotesize 1.2}  \psfrag{1.4}[][]{\footnotesize 1.4}  \psfrag{1.5}[][]{\footnotesize 1.5}  \psfrag{1.6}[][]{\footnotesize 1.6}  \psfrag{0.8}[][]{\footnotesize 1.8}
\psfrag{2.5}[][]{\footnotesize 2.5}  \psfrag{1000}[][]{\footnotesize 1000}\psfrag{2000}[][]{\footnotesize 2000}\psfrag{3000}[][]{\footnotesize 3000}\psfrag{4000}[][]{\footnotesize 4000}
\psfrag{5000}[][]{\footnotesize 5000}
\centering
\hspace*{-0.8in}\subfloat[$h(t)$ vs.~$t$.\label{fig:minimumTimeToClimbAltitude}]{\epsfig{figure=minimumTimeToClimbAltitude.eps,height=2.7in}}~~~\subfloat[$h(t)$ vs.~$v(t)$. \label{fig:minimumTimeToClimbAltitudevsSpeed}]{\epsfig{figure=minimumTimeToClimbAltitudevsSpeed.eps,height=2.7in}}

\hspace*{-0.8in}\subfloat[$\gamma(t)$ vs.~$t$.\label{fig:minimumTimeToClimbFlightPathAngle}]{\epsfig{figure=minimumTimeToClimbFlightPathAngle.eps,height=2.7in}}~~~\subfloat[$\alpha$ vs.~$t$. \label{fig:minimumTimeToClimbAngleofAttack}]{\epsfig{figure=minimumTimeToClimbAngleofAttack.eps,height=2.7in}}

   \subfloat[Mesh Refinement History.\label{fig:minimumTimeToClimbMeshRefinement}]{\epsfig{figure=minimumTimeToClimbMeshRefinement.eps,height=2.7in}}

   \caption{Solution to Minimum Time-to-Climb Problem Using $\mathbb{GPOPS-II}$ with the NLP Solver SNOPT and a Mesh  Refinement Tolerance of $10^{-6}$. \label{fig:minimumTimeToClimbSolution}}
\end{figure}

\begin{table}[h]
\centering
 \caption{Relative Error Estimate vs.~Mesh Refinement Iteration for Minimum Time-to-Climb Problem. \label{tab:minimumTimeToClimbMeshRefinement}}
\small
\begin{tabular}{|c|c|} \hline
{\bf Mesh Refinement Iteration} & {\bf Relative Error Estimate} \\\hline\hline
 1 & $5.776\times 10^{-3}$ \\\hline
 2 & $2.3717\times 10^{-3}$ \\\hline
 3 & $3.0679\times 10^{-5}$ \\\hline
 4 & $6.2216\times 10^{-6}$ \\\hline
 5 & $8.861\times 10^{-6}$ \\\hline
 6 & $2.3224\times 10^{-6}$ \\\hline
 7 & $1.3708\times 10^{-6}$ \\\hline
 8 & $3.8553\times 10^{-6}$ \\\hline
 9 & $5.1621\times 10^{-6}$ \\\hline
 10 & $7.0515\times 10^{-6}$ \\\hline
 11 & $2.5598\times 10^{-6}$ \\\hline
 12 & $1.0775\times 10^{-6}$ \\\hline
 13 & $7.8122\times 10^{-7}$ \\\hline
\end{tabular}
\end{table}

\clearpage

\subsection*{Dynamic Soaring Problem}

The following optimal control problem considers optimizing the motion
of a hang glider in the presence of known wind force.  The probem was
originally described in Ref.~\cite{Zhao2} and the problem considered
here is identical to that of Ref.~\cite{Zhao2}.  The objective is to
minimize the average wind gradient slope $\beta$, that is, minimize
\begin{equation}
  J = \beta
\end{equation}
subject to the hang glider dynamics
\begin{equation}
  \begin{array}{lclclcl}
     \dot{x}& =& v \cos \gamma \sin \psi + W_x &,&
     m\dot{v}& =& -D-mg \sin\gamma -m\dot{W}_x \cos \gamma \sin \psi \\
     \dot{y}& =& v \cos \gamma \cos \psi &,&
     mv\dot{\gamma}& =& L \cos \sigma -mg \cos \gamma + m\dot{W}_x \sin\gamma \sin \psi, \\
     \dot{h}& =& v \sin \gamma &,&
     mv\cos\gamma\dot{\psi}& =& L \sin \sigma -m \dot{W}_x \cos \psi \\
  \end{array}
\end{equation}
and the boundary conditions
\begin{equation}\label{orbit_bcs}
 \begin{array}{lcl}
   (x(0),y(0),h(0))& =& (x(t_f),y(t_f),h(t_f)) = (0,0,0), \\
   (v(t_f)-v(0),\gamma(t_f)-\gamma(0),\psi(t_f)+2\pi-\psi(0)) &= &(0,0,0),
 \end{array}
\end{equation} 
where $W_x$ is the wind  component along the East direction, $m$ is
the glider mass, $v$ is the air-relative speed, $\psi$ is the azimuth
angle  (measured clockwise from the North), $\gamma$ is the
air-relative flight path angle, $h$ is the altitude, $(x,y)$ are (East, North)
position,  $\sigma$ is the glider bank angle, $D$ is the drag force,
and $L$ is the lift force.  The drag and lift forces are computed
using a standard drag polar aerodynamic model
\begin{equation}
 \begin{array}{lcl}
   D = & q S C_D, \\
   L = & q S C_L,
 \end{array}
\end{equation} 
where $q=\rho v^2/2$ is the dynamic pressure, $S$ is the vehicle
reference area, $C_D=C_{D0}+KC_L^2$ is the coefficient of drag, and 
$C_L$ is the coefficient of lift (where $0 \leq C_L \leq C_{L,\max}$).
The constants for this problem ar taken verbatim from
Ref.~\citen{Zhao2} and are given as $C_{D0}=0.00873$, $K=0.045$, and
$C_{L,\max}=1.5$. Finally, it is noted that $C_L$ and $\sigma$ are the
controls.

This example was posed in English units, but was solved using the 
automatic scaling procedure in $\mathbb{GPOPS-II}$ with the NLP solver
IPOPT using second sparse finite-difference approximations for the NLP
derivatives and with a mesh refinement tolerance of $10^{-7}$.  The
code used to solve this problem is shown below and the solution to
this problem is shown in Fig.~\ref{fig:dynamicSoaringSolution}. 
\begin{shadedframe}
\verbatiminput{../examples/dynamicSoaring/dynamicSoaringMain.m}
\verbatiminput{../examples/dynamicSoaring/dynamicSoaringContinuous.m}
\verbatiminput{../examples/dynamicSoaring/dynamicSoaringEndpoint.m}
\end{shadedframe} 

\begin{figure}[h]
\psfragscanon
\psfrag{Time (s)}[][]{\footnotesize$t$ (s)}\psfrag{Altitude (ft)}[][]{\footnotesize$h(t)$ (ft)}\psfrag{X (ft)}[][]{\footnotesize$x(t)$ (ft)}\psfrag{Y (ft)}[][]{\footnotesize$y(t)$ (ft)}
\psfrag{Speed (ft/s)}[][]{\footnotesize$v(t)$ (ft/s)}\psfrag{CL (dimensionless)}[][]{\footnotesize$C_L(t)$ (dimensionless)}\psfrag{Bank Angle (deg)}[][]{\footnotesize$\sigma(t)$ (deg)}
\psfrag{Flight Path Angle (deg)}[][]{\footnotesize$\gamma(t)$ (deg)}\psfrag{Azimuth (deg)}[][]{\footnotesize$\psi(t)$ (deg)}
\psfrag{Angle of Attack (deg)}[][]{\footnotesize$\alpha(t)$ (deg)}\psfrag{Mesh Iteration}[][]{\footnotesize Mesh Refinement Iteration}
\psfrag{Mesh Point Location (Fraction of Interval)}[][]{\footnotesize Mesh Point Locations (Scaled to Interval $[0,t/t_f]$)} 
\psfrag{-500}[][]{\footnotesize -500}\psfrag{-1000}[][]{\footnotesize -1000}
\psfrag{-50}[][]{\footnotesize -50}\psfrag{-100}[][]{\footnotesize -100}\psfrag{-150}[][]{\footnotesize -150}\psfrag{-200}[][]{\footnotesize -200}
\psfrag{-250}[][]{\footnotesize -250}\psfrag{-300}[][]{\footnotesize -300}\psfrag{-350}[][]{\footnotesize -350}\psfrag{-400}[][]{\footnotesize -400}
\psfrag{16.5}[][]{\footnotesize 16.5}\psfrag{17}[][]{\footnotesize 17}\psfrag{17.5}[][]{\footnotesize 17.5}\psfrag{-5}[][]{\footnotesize -5}
\psfrag{-10}[][]{\footnotesize -10}\psfrag{-20}[][]{\footnotesize -20}\psfrag{-30}[][]{\footnotesize -30}\psfrag{-40}[][]{\footnotesize -40}
\psfrag{-50}[][]{\footnotesize -50}\psfrag{-60}[][]{\footnotesize -60}\psfrag{-70}[][]{\footnotesize -70}\psfrag{-80}[][]{\footnotesize -80}
\psfrag{-6}[][]{\footnotesize -6}\psfrag{-4}[][]{\footnotesize -4}\psfrag{-2}[][]{\footnotesize -2}\psfrag{2}[][]{\footnotesize 2}\psfrag{4}[][]{\footnotesize 4}
\psfrag{0}[][]{\footnotesize 0}\psfrag{5}[][]{\footnotesize 5}\psfrag{10}[][]{\footnotesize 10}\psfrag{15}[][]{\footnotesize 15}\psfrag{20}[][]{\footnotesize 20}
\psfrag{25}[][]{\footnotesize 25}\psfrag{30}[][]{\footnotesize 30}\psfrag{35}[][]{\footnotesize 35}\psfrag{40}[][]{\footnotesize 40}\psfrag{45}[][]{\footnotesize 45}
\psfrag{50}[][]{\footnotesize 50}\psfrag{55}[][]{\footnotesize 55}\psfrag{60}[][]{\footnotesize 60}\psfrag{65}[][]{\footnotesize 65}
\psfrag{70}[][]{\footnotesize 70}\psfrag{75}[][]{\footnotesize 75}\psfrag{80}[][]{\footnotesize 80}\psfrag{85}[][]{\footnotesize 85}\psfrag{90}[][]{\footnotesize 90}
\psfrag{95}[][]{\footnotesize 95}\psfrag{100}[][]{\footnotesize 100}\psfrag{150}[][]{\footnotesize 150}\psfrag{200}[][]{\footnotesize 200}
\psfrag{120}[][]{\footnotesize 120}\psfrag{160}[][]{\footnotesize 160}\psfrag{200}[][]{\footnotesize 200}\psfrag{240}[][]{\footnotesize 240}
\psfrag{250}[][]{\footnotesize 250}\psfrag{300}[][]{\footnotesize 300}\psfrag{350}[][]{\footnotesize 350}\psfrag{400}[][]{\footnotesize 400}
\psfrag{600}[][]{\footnotesize 600}\psfrag{800}[][]{\footnotesize 800}
\psfrag{500}[][]{\footnotesize 500}\psfrag{1000}[][]{\footnotesize 1000}\psfrag{1500}[][]{\footnotesize 1500}\psfrag{2000}[][]{\footnotesize 2000}
\psfrag{2500}[][]{\footnotesize 2500}\psfrag{3000}[][]{\footnotesize 3000}\psfrag{3500}[][]{\footnotesize 3500}\psfrag{4000}[][]{\footnotesize 4000}
\psfrag{4500}[][]{\footnotesize 4500}\psfrag{5000}[][]{\footnotesize 5000}\psfrag{5500}[][]{\footnotesize 5500}\psfrag{6000}[][]{\footnotesize 6000}
\psfrag{6500}[][]{\footnotesize 6500}\psfrag{7000}[][]{\footnotesize 7000}\psfrag{7500}[][]{\footnotesize 7500}\psfrag{8000}[][]{\footnotesize 8000}
\psfrag{6}[][]{\footnotesize 6}  \psfrag{7}[][]{\footnotesize 7}  \psfrag{8}[][]{\footnotesize 8}  \psfrag{9}[][]{\footnotesize 9}  \psfrag{10}[][]{\footnotesize 10}  \psfrag{11}[][]{\footnotesize 11}
\psfrag{12}[][]{\footnotesize 12}  \psfrag{13}[][]{\footnotesize 13}
\psfrag{-1.5}[][]{\footnotesize -1.5} \psfrag{-1}[][]{\footnotesize -1}
\psfrag{-0.5}[][]{\footnotesize -0.5} \psfrag{-0.2}[][]{\footnotesize -0.2}\psfrag{0.1}[][]{\footnotesize 0.1}
\psfrag{0.2}[][]{\footnotesize 0.2}\psfrag{0.3}[][]{\footnotesize 0.3}  \psfrag{0.4}[][]{\footnotesize 0.4}  \psfrag{0.5}[][]{\footnotesize 0.5}\psfrag{0.6}[][]{\footnotesize 0.6}  \psfrag{0.8}[][]{\footnotesize 0.8}
\psfrag{1.2}[][]{\footnotesize 1.2}  \psfrag{1.4}[][]{\footnotesize 1.4}  \psfrag{1.5}[][]{\footnotesize 1.5}  \psfrag{1.6}[][]{\footnotesize 1.6}  \psfrag{0.8}[][]{\footnotesize 1.8}
\psfrag{2.5}[][]{\footnotesize 2.5}  \psfrag{1000}[][]{\footnotesize 1000}\psfrag{2000}[][]{\footnotesize 2000}\psfrag{3000}[][]{\footnotesize 3000}\psfrag{4000}[][]{\footnotesize 4000}
\psfrag{5000}[][]{\footnotesize 5000}
\centering
\hspace*{-0.5in}\subfloat[$h(t)$ vs.~$x(t)$ and $y(t)$.\label{fig:dynamicSoaringXYZ}]{\epsfig{figure=dynamicSoaringxyh.eps,height=2.4in}}~~~\subfloat[$v(t)$ vs.~$t$. \label{fig:dynamicSoaringSpeed}]{\epsfig{figure=dynamicSoaringSpeed.eps,height=2.7in}}

\hspace*{-0.5in}\subfloat[$\gamma(t)$ vs.~$t$.\label{fig:dynamicSoaringFlightPathAngle}]{\epsfig{figure=dynamicSoaringFlightPathAngle.eps,height=2.7in}}~~~\subfloat[$\psi$ vs.~$t$. \label{fig:dynamicSoaringAzimuth}]{\epsfig{figure=dynamicSoaringAzimuth.eps,height=2.7in}}

\hspace*{-0.5in}\subfloat[$C_L(t)$ vs.~$t$.\label{fig:dynamicSoaringCL}]{\epsfig{figure=dynamicSoaringCL.eps,height=2.7in}}~~~\subfloat[$\sigma(t)$ vs.~$t$.\label{fig:dynamicSoaringBankAngle}]{\epsfig{figure=dynamicSoaringBankAngle.eps,height=2.7in}}

   \caption{Solution to Dynamic Soaring Problem Using $\mathbb{GPOPS-II}$ with the NLP Solver IPOPT and a Mesh  Refinement Tolerance of $10^{-6}$. \label{fig:dynamicSoaringSolution}}
\end{figure}

\clearpage

\subsection{Two-Strain Tuberculosis Optimal Control Problem}

Quoting from Ref.~\citen{Jung02}, ``[Past]
models [for Tuberculosis (TB)] did not account for time dependent control
strategies$\ldots$In this article we consider (time dependent) optimal
control strategies associated with case holding and case finding based
on a two-strain TB model$\ldots$Our objective functional balances the
effect of minimizing the cases of latent and infectious drug-resistant
TB and minimizing the cost of implementing the control treatments.''
The two-strain tuberculosis optimal control problem considered in 
Ref.~\citen{Jung02} is formulated as follows.  Minimize the objective
functional
\begin{equation}\label{TB-cost}
  J = \int_0^{t_f} \left[\scriptsize  L_2 + I_2 + \frac{1}{2}  B_1 u_1^2 + B_2  u_2^2\right]dt 
\end{equation}
subject to the dynamic constraints
\begin{equation}\label{TB-dynamics}
 \begin{array}{lcl}
    \dot{S}(t) & = & \Lambda - \beta_1 S\frac{I_1 (t)}{N} - \beta^*  S\frac{I_2 (t)}{N} - \mu S (t), \\
    \dot{L}_1 (t) & = &  \beta_1 S (t)\frac{I_1 (t)}{N} - (\mu + k_1)L_1 (t) - u_1  r_1L_1 (t) \\
    & + & (1-u_2 (t))p r_2 I_1 (t) +\beta_2 T (t) \frac{I_1 (t)}{N} - \beta^* L_1 (t) \frac{I_2 (t)}{N}, \\
    \dot{I}_1 (t) & = & k_1 L_1 (t) - (\mu + d_1) I_1 (t) - r_2 I_1 (t), \\
    \dot{L}_2 (t) & = &  (1-u_2 (t)) q r_2 I_1 (t) - (\mu + k_2) L_2 (t) \\
    & + & \beta^*  (S (t)+L_1 (t) + T (t))\frac{I_2 (t)}{N}, \\
    \dot{I}_2 (t) & = &  k_2 L_2 (t) - (\mu+d_2)I_2 (t), \\
    \dot{T}(t) & = &  u_1 (t) r_1 L_1 (t) - (1-(1-u_2 (t)))(p+q)r_2 I_1 (t) \\
    & - & \beta_2 T (t) \frac{I_1 (t)}{N} - \beta^* T (t) \frac{I_2 (t)}{N} - \mu T (t),\\
    0 & = & S + L_1 + I_1 + L_2 + I_2 + T - N,
  \end{array}
\end{equation}
and the initial conditions
\begin{equation}\label{TB-ICs}
  (S(0),L_1(0),I_1(0),L_2(0),I_2(0),T(0))=(S_0,L_{10},I_{10},L_{20},I_{20},T_0), 
\end{equation}
where etails of the model can be found in Ref.~\citen{Jung02} (and are
also provided in the $\mathbb{GPOPS-II}$ code shown below).  The optimal 
control problem of Eqs.~(\ref{TB-cost})--(\ref{TB-ICs}) is solved
using $\mathbb{GPOPS-II}$ with the NLP solver SNOPT with a mesh
refinement accuracy tolerance of $10^{-6}$.   The code used to solve
this example is given below and the solution is shown in
Figs.~\ref{fig:tuberculosisState} and \ref{fig:tuberculosisControl}.
\begin{shadedframe}
\verbatiminput{../examples/TuberculosisModel/tuberculosisMain.m}
\verbatiminput{../examples/TuberculosisModel/tuberculosisContinuous.m}
\verbatiminput{../examples/TuberculosisModel/tuberculosisEndpoint.m}
\end{shadedframe}
\begin{figure}[h]
\psfragscanon
\psfrag{Time (s)}[][]{$t$ (s)}\psfrag{S x 1000}[][]{$S(t)\times 1000$}\psfrag{L1 x 1000}[][]{$L_1(t)\times 1000$}\psfrag{L2 x 1000}[][]{$L_2(t)\times 1000$}\psfrag{I1 x 1000}[][]{$I_1(t)\times 1000$}
\psfrag{I2 x 1000}[][]{$I_2(t)\times 1000$}\psfrag{T}[][]{$T(t)$}
\psfrag{1}[][]{\footnotesize 1}\psfrag{3}[][]{\footnotesize 3}\psfrag{5}[][]{\footnotesize 5}\psfrag{7}[][]{\footnotesize 7}\psfrag{9}[][]{\footnotesize 9}
\psfrag{2}[][]{\footnotesize 2}\psfrag{4}[][]{\footnotesize 4}\psfrag{6}[][]{\footnotesize 6}\psfrag{8}[][]{\footnotesize 8}\psfrag{10}[][]{\footnotesize 10}
\psfrag{12}[][]{\footnotesize 12}\psfrag{14}[][]{\footnotesize 14}\psfrag{16}[][]{\footnotesize 16}\psfrag{18}[][]{\footnotesize 18}\psfrag{20}[][]{\footnotesize 20}
\psfrag{100}[][]{\footnotesize 100}\psfrag{200}[][]{\footnotesize 200}\psfrag{300}[][]{\footnotesize 300}\psfrag{400}[][]{\footnotesize 400}\psfrag{500}[][]{\footnotesize 500}
\psfrag{600}[][]{\footnotesize 600}\psfrag{700}[][]{\footnotesize 700}\psfrag{800}[][]{\footnotesize 800}\psfrag{900}[][]{\footnotesize 900}\psfrag{1000}[][]{\footnotesize 1000}
\psfrag{150}[][]{\footnotesize 150}\psfrag{250}[][]{\footnotesize 250}\psfrag{350}[][]{\footnotesize 350}\psfrag{450}[][]{\footnotesize 450}\psfrag{550}[][]{\footnotesize 550}
\centering
\hspace*{-0.5in}\subfloat[$S(t)$ vs.~$t$.\label{fig:tuberculosisS}]{\epsfig{figure=tuberculosisS.eps,height=2.7in}}~~~\subfloat[$L_1(t)$ vs.~$t$. \label{fig:tuberculosisL1}]{\epsfig{figure=tuberculosisL1.eps,height=2.7in}}

\hspace*{-0.5in}\subfloat[$L_2(t)$ vs.~$t$.\label{fig:tuberculosisL2}]{\epsfig{figure=tuberculosisL2.eps,height=2.7in}}~~~\subfloat[$I_1$ vs.~$t$. \label{fig:tuberculosisI1}]{\epsfig{figure=tuberculosisI1.eps,height=2.7in}}

\hspace*{-0.5in}\subfloat[$I_2(t)$ vs.~$t$.\label{fig:tuberculosisI2}]{\epsfig{figure=tuberculosisI2.eps,height=2.7in}}~~~\subfloat[$T(t)$ vs.~$t$.\label{fig:tuberculosisT}]{\epsfig{figure=tuberculosisT.eps,height=2.7in}}

   \caption{Optimal State for Tuberculosis Optimal Control Problem Using $\mathbb{GPOPS-II}$ with the NLP Solver SNOPT and a Mesh  Refinement Tolerance of $10^{-6}$. \label{fig:tuberculosisState}}
\end{figure}

\begin{figure}[h]
\psfragscanon
\psfrag{Time (s)}[][]{$t$ (s)}\psfrag{U1}[][]{$u_1(t)$}\psfrag{U2}[][]{$u_2(t)$}
\psfrag{0.1}[][]{\footnotesize 0.1}\psfrag{0.3}[][]{\footnotesize 0.3}\psfrag{0.5}[][]{\footnotesize 0.5}\psfrag{0.7}[][]{\footnotesize 0.7}\psfrag{0.9}[][]{\footnotesize 0.9}
\psfrag{0.2}[][]{\footnotesize 0.2}\psfrag{0.4}[][]{\footnotesize 0.4}\psfrag{0.6}[][]{\footnotesize 0.6}\psfrag{0.8}[][]{\footnotesize 0.8}\psfrag{1}[][]{\footnotesize 1}
\psfrag{-0.1}[][]{\footnotesize -0.1}\psfrag{-0.3}[][]{\footnotesize -0.3}\psfrag{-0.5}[][]{\footnotesize -0.5}\psfrag{-0.7}[][]{\footnotesize -0.7}\psfrag{-0.9}[][]{\footnotesize -0.9}
\psfrag{-0.2}[][]{\footnotesize -0.2}\psfrag{-0.4}[][]{\footnotesize -0.4}\psfrag{-0.6}[][]{\footnotesize -0.6}\psfrag{-0.8}[][]{\footnotesize -0.8}\psfrag{-1}[][]{\footnotesize -1}
\psfrag{1}[][]{\footnotesize 1}\psfrag{3}[][]{\footnotesize 3}\psfrag{5}[][]{\footnotesize 5}\psfrag{7}[][]{\footnotesize 7}\psfrag{9}[][]{\footnotesize 9}
\psfrag{2}[][]{\footnotesize 2}\psfrag{4}[][]{\footnotesize 4}\psfrag{6}[][]{\footnotesize 6}\psfrag{8}[][]{\footnotesize 8}\psfrag{10}[][]{\footnotesize 10}
\centering

\hspace*{-0.5in}\subfloat[$u_1(t)$ vs.~$t$.\label{fig:tuberculosisU1}]{\epsfig{figure=tuberculosisU1.eps,height=2.7in}}~~~\subfloat[$u_2(t)$ vs.~$t$.\label{fig:tuberculosisU2}]{\epsfig{figure=tuberculosisU2.eps,height=2.7in}}

   \caption{Optimal Control for Tuberculosis Optimal Control Problem Using $\mathbb{GPOPS-II}$ with the NLP Solver SNOPT and a Mesh  Refinement Tolerance of $10^{-6}$. \label{fig:tuberculosisControl}}
\end{figure}

\clearpage

\section{Concluding Remarks}

While the authors have put for the effort to make $\mathbb{GPOPS-II}$ a
user-friendly software, it is important to understand several aspects
of computational optimal control in order to make $\mathbb{GPOPS-II}$ easier to
use.  First, it is {\em highly} recommended that the user scale a
problem manually using insight from the physics/mathematics of the
problem because the automatic scaling procedure is by no means
foolproof.  Second, the particular parameterization of a problem can
make all the difference with regard to obtaining a solution in a
reliable manner.  Finally, even if the NLP solver returns the result that the
optimality conditions have been satisfied, it is important to verify
the solution.  In short, a great deal of time in solving optimal
control problems is spent in formulation and analysis.

\bibliographystyle{aiaa}
\bibliography{master}

\end{document}
